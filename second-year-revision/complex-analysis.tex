\documentclass[10pt,fleqn]{article}

\author{Timothy Hosgood}
\title{Complex Analysis}
\pagestyle{headings}
\usepackage{amsmath}
\usepackage{amssymb}
\usepackage{amsthm}
\usepackage{enumerate}
\usepackage{embedfile}

\newcommand{\diff}{\,\mathrm{d}}
\newcommand{\met}{\mathrm{d}}
\newcommand{\comps}{\mathbb{C}}
\newcommand{\reals}{\mathbb{R}}
\newcommand{\eps}{\varepsilon}
\newcommand{\vc}[1]{\boldsymbol{#1}}
\newcommand{\re}{\mathrm{Re}}
\newcommand{\im}{\mathrm{Im}}
\newcommand{\res}{\mathrm{res}}
\theoremstyle{definition} \newtheorem{defn}{Definition}[section]
\theoremstyle{plain}      \newtheorem{thm}[defn]{Theorem}
\theoremstyle{definition} \newtheorem{prop}[defn]{Proposition}
\theoremstyle{plain}      \newtheorem{lem}[defn]{Lemma}
\theoremstyle{definition} \newtheorem{cor}[defn]{Corollary}
\theoremstyle{definition} \newtheorem{ex}[defn]{Example}
\theoremstyle{definition} \newtheorem{rem}[defn]{Remark}

\begin{document}
\embedfile{\jobname.tex}
\maketitle
\begin{abstract}
    These notes are based entirely on lectures given by Richard Earl to second-year undergraduates at the University of Oxford in the year 2013/14.
\end{abstract}
\tableofcontents


\section{Geometry of the complex plane}

\begin{prop}[Circles and lines]
    Let $A,C\in\reals,B\in\comps$.
    Then the equation
    \begin{equation}
        Az\bar{z}+B\bar{z}+\bar{B}z+C=0
    \end{equation}
    represents
    \begin{enumerate}[(i)]
        \item $A=0$:
        a line
        \item $A\neq0$ and $|B|^2\geq AC$:
        a circle, centre $\frac{-B}{A}$, radius $\frac{1}{|A|}\sqrt{|B|^2-AC}$
        \item otherwise:
        no solutions
    \end{enumerate}

    Further, all circles and lines can be represented in this way.
\end{prop}

\begin{proof}
    If $A\neq0$ then we can rewrite the equation as
    \[
        \left|z+\frac{B}{A}\right|^2=
        \frac{|B|^2-AC}{A^2}
    \]
    which is the equation of the circle, as required, unless $|B|^2<AC$, in which case there are no solutions.

    If $A=0$ then we can rewrite the equation as
    \[
        B\bar{z}+\bar{B}z+C=0
    \]
    Letting $B=u+iv$ and $z=x+iy$, this is
    \[
        2ux+2vy+C=0
    \]
    which is the equation of a line.

    Clearly any line is of the above form, and any circle $|z-a|^2=r^2$ can be put into the above form too.
\end{proof}

\begin{thm}[Apollonius' Theorem]
    Let $k>0$ with $k\neq1$, and $\alpha,\beta\in\comps$ with $\alpha\neq\beta$.
    Then the locus of points satisfying the equation
    \begin{equation}
        |z-\alpha|=
        k|z-\beta|
    \end{equation}
    is a circle centre $c$, radius $r$, where
    \[
        c=\frac{k^2\beta-\alpha}{k^2-1},\qquad
        r=\frac{k|\alpha-\beta|}{|k^2-1|}
    \]

    Further, $\alpha$ and $\beta$ are inverse points.
    That is, $\alpha$ and $\beta$ are collinear with $c$, and $|c-\alpha||c-\beta|=r^2$.
\end{thm}

\begin{proof}
    Squaring the equation gives
    \[
        (z-\alpha)(\bar{z}-\bar{\alpha})=
        k^2(z-\beta)(\bar{z}-\bar{\beta})
    \]
    which we can rearrange to get
    \[
        \left|z-\left(\frac{k^2\beta-\alpha}{k^2-1}\right)\right|^2=
        \frac{k^2|\alpha-\beta|^2}{(k^2-1)^2}
    \]

    Note that $c$ is collinear with $\alpha$ and $\beta$ as
    \[
        c=
        \alpha+\frac{k^2}{k^2-1}(\beta-\alpha)=
        \beta+\frac{1}{k^2-1}(\beta-\alpha)
    \]
    and from these we also see that
    \[
        |c-\alpha||c-\beta|=r^2
    \]
\end{proof}


\section{Holomorphic functions}

\begin{defn}[Discs in the complex plane]
    Let $a\in\comps$ and $r>0$.
    Then we define
    \begin{equation}
        \begin{array}{rrcl}
            \text{open disc:} & D(a,r) & := & \{z\in\comps \colon |z-a|<r\}\\
            \text{closed disc:} & \bar{D}(a,r) & := & \{z\in\comps \colon |z-a|\leq r\}\\
            \text{punctured disc:} & D'(a,r) & := & \{z\in\comps \colon 0<|z-a|<r\}
        \end{array}
    \end{equation}
\end{defn}

\begin{defn}[Differentiable and holomorphic functions]
    Let $U$ be an open subset of $\comps$ and $f:U\to\comps$.
    Let $z_0\in U$.
    Then $f$ is \emph{differentiable} at $z_0$ if
    \begin{equation}
        \lim_{h\to0} \frac{f(z_0+h)-f(z_0)}{h}
    \end{equation}
    exists, in which case we write $f'(z_0)$ to mean this limit.

    Precisely, this means that
    \[
        \exists L\in\comps,~
        \forall\eps>0,~
        \exists\delta>0,~
        \left[h\in D'(0,\delta)\implies\left|\frac{f(z_0+h)-f(z_0)}{h}-L\right|<\eps\right]
    \]

    Further, $f$ is \emph{holomorphic} on $U$ if it is differentiable at every $x_0\in U$.
\end{defn}

\begin{thm}[Differentiable implies continuous]
    Let $f:U\to\comps$ be differentiable at $x_0\in U$.
    Then $f$ is continuous at $x_0$.
\end{thm}

\begin{proof}
    \begin{align*}
        \lim_{h\to0} [f(z_0+h)-f(z_0)]
        &=
        \lim_{h\to0} \left[h\cdot\frac{f(z_0+h)-f(z_0)}{h}\right]\\
        &=
        (\lim_{h\to0} h)(\lim_{h\to0}\left[\frac{f(z_0+h)-f(z_0)}{h}\right])\\
        &=
        (\lim_{h\to0} h)(f'(z_0))\\
        &=
        0
    \end{align*}
    So
    \[
        \lim_{z\to z_0}f(z)=
        f(z_0)
    \]
    as required.
\end{proof}

\begin{rem}
    Being differentiable is a stronger requirement than just $\frac{\partial f}{\partial x}$ and $\frac{\partial f}{\partial y}$ existing; $h$ must be able to approach 0 from any direction, and not just along either axis.
\end{rem}

\begin{thm}[Cauchy-Riemann Equations]
    Let $f:U\to\comps$ be differentiable at $z_0\in U$.
    For $z\in U$ let $x=\re z, y=\im z$, and let $u=\re f,v=\im f$.
    Then, at $z_0$,
    \[
        u_x=v_y,\quad
        u_y=-v_x
    \]
\end{thm}

\begin{proof}
    As $f$ is differentiable at $z_0\in U$, in particular the limits as $h\to0$ along the real axis and along the imaginary axis in the definition of $f'(z_0)$ must be equal.
    Let $z_0=x_0+iy_0$.
    Then
    \begin{align*}
        f'(z_0)
        &=
        \underset{h\in\reals}{\lim_{h\to0}}\left[\frac{f(z_0+h)-f(z_0)}{h}\right]\\
        &=
        \lim_{h\to0}\left[\frac{[u(x_0+h,y_0)+iv(x_0+h,y_0)]-[u(x_0,y_0)+iv(x_0,y_0)]}{h}\right]\\
        &=
        \lim_{h\to0}\left[\left(\frac{u(x_0+h,y_0)-u(x_0,y_0)}{h}\right)+i\left(\frac{v(x_0+h,y_0)-v(x_0,y_0)}{h}\right)\right]\\
        &=
        u_x(x_0,y_0)+iv_x(x_0,y_0)
    \end{align*}
    and
    \begin{align*}
        f'(z_0)
        &=
        \underset{h\in\reals}{\lim_{h\to0}}\left[\frac{f(z_0+ih)-f(z_0)}{ih}\right]\\
        &=
        \lim_{h\to0}\left[\frac{[u(x_0,y_0+h)+iv(x_0,y_0+h)]-[u(x_0,y_0)+iv(x_0,y_0)]}{ih}\right]\\
        &=
        \lim_{h\to0}\left[\frac{1}{i}\left(\frac{u(x_0,y_0+h)-u(x_0,y_0)}{h}\right)+\left(\frac{v(x_0,y_0+h)-v(x_0,y_0)}{h}\right)\right]\\
        &=
        -iu_y(x_0,y_0)+v_y(x_0,y_0)
    \end{align*}

    Hence, comparing real and imaginary parts gives the result required.
\end{proof}

\begin{defn}[Wirtinger derivatives]
    The \emph{Wirtinger derivatives}, for a complex-valued function on $\comps$, are defined as
    \begin{equation}
        \frac{\partial}{\partial\bar{z}}=\frac{1}{2}\left(\frac{\partial}{\partial x}+i\frac{\partial}{\partial y}\right),\quad
        \frac{\partial}{\partial z}=\frac{1}{2}\left(\frac{\partial}{\partial x}-i\frac{\partial}{\partial y}\right)
    \end{equation}
\end{defn}

\begin{thm}[Equivalance of Cauchy-Riemann equations]
    Let $f:U\to\comps$ be a complex-valued function on an open set $U\subseteq\comps$.
    The Cauchy-Riemann equations are equivalent to
    \[
        \frac{\partial f}{\partial\bar{z}}=0
    \]
\end{thm}

\begin{proof}
    Let $u=\re f,v=\im f$.
    Then
    \[
        \frac{\partial f}{\partial\bar{z}}=
        \frac{1}{2}[(u_x-v_y)+i(u_y+v_x)]
    \]

    Comparing real and imaginary parts, we see that
    \[
        \frac{\partial f}{\partial\bar{z}}=0\Longleftrightarrow
        \left\{
        \begin{array}{l}
            u_x = v_y\\
            u_y = -v_x
        \end{array}
        \right.
    \]
\end{proof}

\begin{prop}[Goursat]
    Let $f=u+iv:U\to\comps$.
    Suppose that $u_x,u_y,v_x,v_y$ exist, are continuous, and satisfy the Cauchy-Riemann equations at $z_0\in U$.
    Then $f$ is differentiable at $z_0$.
\end{prop}

\begin{proof}
    Let $z_0=x+iy$ and $\eps>0$ be such that $D(z_0,\eps)\subseteq U$.
    Let $h=p+iq$ with $|h|<\eps$.
    Then, by the Mean-value Theorem, with each $\theta_i\in(0,1)$, we have
    \begin{align*}
        \frac{f(z_0+h)-f(z_0)}{h}
        &=
        \frac{p}{h}\left(\frac{u(x+p,y+q)-u(x,y+q)+iv(x+p,y+q)-iv(x,y+q)}{p}\right)\\
        &\quad+
        \frac{q}{h}\left(\frac{u(x,y+q)-u(x,y)+iv(x,y+q)-ev(x,y)}{q}\right)\\
        &=
        \frac{p}{h}(u_x(x+\theta_1p,q)+iv_x(x+\theta_2p,q))\\
        &\quad+
        \frac{q}{h}(u_y(x,y+\theta_3q)+iv_y(x,y+\theta_4q))
    \end{align*}

    Using the Cauchy-Riemann equations, this can be rewritten as
    \begin{align*}
        \frac{f(z_0+h)-f(z_0)}{h}
        &=
        \frac{[pu_x(x+\theta_1p,y+q)+iqu_x(x,y+\theta_4q)]}{p+iq}\\
        &\quad+
        \frac{i[pv_x(x+\theta_2p,y+q)+iqv_x(x,y+\theta_3q)]}{p+iq}
    \end{align*}

    Finally, using the continuity of $u_x$ and $v_x$, we see that
    \[
        f'(z_0)=
        \lim_{h\to0}\left(\frac{f(z_0+h)-f(z_0)}{h}\right)=
        u_x(x,y)+iv_x(x,y)
    \]
    exists.
\end{proof}

\begin{cor}[Harmonic parts of $f$]
    Let $f=u+iv$ be holomorphic on an open set $U$.
    Then $u$ and $v$ are harmonic functions (that is, they satisfy Laplace's equation).
\end{cor}

\begin{proof}
    As $f$ is holomorphic, the Cauchy-Riemann equations are satisfied.
    But we have just shown that
    \[
        f'=u_x+iv_x
    \]

    If we assume that the derivative of $f$ is holomorphic (something which we shall soon prove), then we have that these real and imaginary parts of $f'$ must also satisfy the Cauchy-Riemann equations.
    That is,
    \begin{align*}
        (u_x)_x&=(v_x)_y=(-u_y)_y\\
        (v_x)_x&=-(u_x)_y=-(v_y)_y
    \end{align*}
    And as all these second derivatives are continuous (by the assumption that $f'$ is holomorphic), thus
    \begin{align*}
        u_{xx}+u_{yy}&=0\\
        v_{xx}+v_{yy}&=0
    \end{align*}
\end{proof}

\begin{defn}[Harmonic conjugates]
    If $u,v$ are harmonic functions on the open set $U\subseteq\comps$ such that $u+iv$ is holomorphic on $U$, then $v$ is a \emph{harmonic conjugate} of $u$.
\end{defn}


\section{Power series and the complex exponential}

\begin{defn}[Power series]
    Let $a,z\in\comps$ and $(a_n)\subseteq\comps$ be a complex sequence.
    A \emph{power series centered on $a$} is a series of the form
    \begin{equation}
        \sum_{n=0}^{\infty}a_n(z-a)^n
    \end{equation}

    We treat $(a_n)$ as given and $z$ as a complex variable, so that the series, where it converges, defines a function in $z$.
\end{defn}

\begin{thm}[Radius of convergence]
    Given a power series $\sum_{n=0}^{\infty}a_n(z-a)^n$, there exists a unique $R\in[0,\infty)$, called the \emph{radius of convergence} of the series, such that the power series
    \begin{enumerate}[(i)]
        \item converges absolutely when $|z-a|<R$
        \item diverges when $|z-a|>R$
    \end{enumerate}

    Further, within the \emph{disc of convergence} $|z-a|<R$, the power series defines a differentiable function $f(z)$, and term-by-term differentiation is valid within this disc.
    That is,
    \begin{equation}
        f'(z)=
        \sum_{n=1}^{\infty}na_n(z-a)^{n-1}
    \end{equation}
\end{thm}

\begin{prop}[Strong ratio test]
    Given a power series $\sum_{n=0}^{\infty}a_n(z-a)^n$, if
    \begin{equation}
        L:=
        \lim_{n\to\infty}\frac{a_n}{a_{n+1}}
    \end{equation}
    exists then $R=|L|$
\end{prop}

\begin{prop}[$(1-z)^{-1}$]
    The radius of convergence of
    \begin{equation}
        \sum_{n=0}^{\infty}z^n
    \end{equation}
    is 1, and so the series defines a holomorphic function on $D(0,1)$, which converges nowhere on the boundary $|z|=1$.

    Then function it defines is $(1-z)^{-1}$.
\end{prop}

\begin{rem}
    We later prove that any function holomorphic at $a$ has a power series centred at $a$, whose radius of convergence is equal to the distance from $a$ to the nearest singularity.
\end{rem}

\begin{thm}[Complex exponential and trigonometric functions]
    The following power series, defining the complex exponential and trigonometric functions, converges on the entire complex plane:
    \begin{equation}
        \exp z:=\sum_{n=0}^{\infty}\frac{z^n}{n!},\qquad
        \sin z:=\sum_{n=0}^{\infty}\frac{(-1)^nz^{2n+1}}{(2n+1)!},\qquad
        \cos z:=\sum_{n=0}^{\infty}\frac{(-1)^nz^{2n}}{(2n)!}
    \end{equation}

    Further, these functions have the following properties for any $z,w\in\comps,r\in\reals$:
    \begin{enumerate}[(i)]
        \item $\exp'z=\exp z,\sin'z=\cos z$, and $\cos'z=-\sin z$
        \item $\exp(z+w)=(\exp z)(\exp w)$
        \item $\exp(iz)=\cos z+i\sin z$
        \item $\cos z=\frac{\exp(iz)+\exp(-iz)}{2}$, and $\sin z=\frac{\exp(iz)-\exp(-iz)}{2i}$
        \item $\sin^2z+\cos^2z=1$
        \item $\cos$ and $\sin$ have smallest period $2\pi$
        \item $\exp$ has smallest period $2\pi i$
        \item $\exp r=e^r$ (and thus we often write $e^z$ for complex $z$ to mean $\exp z$)
    \end{enumerate}
\end{thm}

\begin{prop}[Principal branch of the complex logarithm]
    \begin{enumerate}[(i)]
        \item Any $z\in\comps\setminus(-\infty,0]$ can be written uniquely as $z=re^{i\theta}$, where $r>0,\theta\in(-\pi,\pi)$.
        \item The function $L:\comps\setminus(-\infty,0]$ given by
        \begin{equation}
            L(z)=\log r+i\theta
        \end{equation}
        satisfies $(\exp\circ L)(z)=z$, and is holomorphic, with $L'(z)=1/z$.
    \end{enumerate}
\end{prop}

\begin{proof}
    \begin{enumerate}[(i)]
        \item This follows from choosing $r=|z|$ and $\theta=\arg z$, which takes a unique prinicipal value in the given range.
        \item First, note that
        \[
            exp(L(z))=
            e^{\log r}e^{i\theta}=
            re^{i\theta}=
            z
        \]
        and that
        \begin{align*}
            u(x,y)
            &=
            \log\sqrt{x^2+y^2}
            =
            \frac{1}{2}\log(x^2+y^2)\\
            v(x,y)
            &=
            \arctan(\frac{y}{x})
        \end{align*}
        and, by differentiating, we can see that the Cauchy-Riemann equations are satisfied.
        Further, $u_x,u_y,v_x$, and $v_y$ are all clearly continuous, so $L$ is holomorphic.
        Also
        \[
            L'(z)=
            u_x+iv_x=
            \frac{x-iy}{x^2+y^2}=
            \frac{1}{x+iy}=
            \frac{1}{z}
        \]

        \emph{N.B. Some care should be taken with the different formulae for $\arg$ depending on the sign of $y$, but all the above follows in a nearly identical fashion.}
    \end{enumerate}
\end{proof}

\begin{cor}[Complex powers]
    Let $\alpha\in\comps$.
    Then
    \begin{equation}
        z^{\alpha}:=
        \exp(\alpha L(z))
    \end{equation}
    defines a holomorphic function on $\comps\setminus(-\infty,0]$, and
    \[
        \frac{\mathrm{d}}{\mathrm{d}z}(z^{\alpha})=
        \alpha z^{\alpha-1}
    \]
\end{cor}

\begin{proof}
    As $L$ and $\exp$ are both holomorphic, then so too is $z^{\alpha}$.
    Then, by the chain rule,
    \begin{align*}
        \frac{\mathrm{d}}{\mathrm{d}z}
        &=
        \frac{\mathrm{d}}{\mathrm{d}z}\exp(\alpha L(z))\\
        &=
        \alpha L'(z)\exp(\alpha L(z))\\
        &=
        \frac{\alpha}{z}\exp(\alpha L(z))\\
        &=
        \alpha\exp(\alpha L(z))\exp(-L(z))\\
        &=
        \alpha\exp((\alpha-1)L(z))\\
        &=
        \alpha z^{\alpha-1}
    \end{align*}
\end{proof}

\begin{rem}
    Note that, for $z,w\in\comps\setminus(-\infty,0]$,
    \[
        L(zw)\neq
        L(z)+L(w)
    \]
\end{rem}

\begin{rem}[Holomorphic branches]
    The function $L(z)$ defined on \\$\comps\setminus(-\infty,0]$ is a \emph{holomorphic branch} of the complex logarithm.
    The other branches of the logarithm on this cut plane are $L(z)+2n\pi i$.
    
    We could also define a logarithm on other cut planes, for example on \\$\comps\setminus\{\text{negative imaginary axis}\}$, by simply taking $\arg z\in(-\pi/2,3\pi/2)$.
\end{rem}

\begin{defn}[Mutilvalued functions]
    Let $\alpha,z\in\comps,z\neq0$.
    Define
    \begin{equation}
        \begin{array}{rcl}
            [\arg z] &=& \{\theta\in\reals \colon z=|z|e^{i\theta}\}\\
            \left[\log z\right] &=& \{w\in\comps \colon e^w=z\}\\
            \left[z^{\alpha}\right] &=& \exp(\alpha[\log z])
        \end{array}
    \end{equation}
\end{defn}

\begin{prop}
    Let $z,w,\alpha,\beta\in\comps$ with $z,w\neq0$.
    Then
    \begin{enumerate}[(i)]
        \item $[\log z]+[\log w]=[\log(zw)]$
        \item $[z^{\alpha}][w^{\alpha}]=[(zw)^{\alpha}]$
        \item $[z^{\alpha}][z^{\beta}]\neq[z^{\alpha+\beta}]$
    \end{enumerate}
\end{prop}


\section{Cauchy's Theorem}

\begin{defn}[Domains]
    A subset $U\subseteq\comps$ is a \emph{domain} if it is non empty, open, and connected.
\end{defn}

\begin{defn}[Paths/Contours]
    A \emph{path} or \emph{contour} $\gamma$ is a continuous and piecewise continuously differentiable function
    \begin{equation}
        \gamma:[a,b]\to\comps
    \end{equation}

    N.B. Sometimes the image of $\gamma$ is also referred to as the path or contour.

    A path $\gamma:[a,b]\to\comps$ is said to be \emph{simple} if $\gamma$ is injective, with the possible exception that $\gamma(a)=\gamma(b)$ is allowed, in which case the path is said to be \emph{closed}.
\end{defn}

\begin{thm}[Jordan Curve Theorem]
    Let $\gamma$ be the image of a simple closed path in $\comps$.
    The complement $\comps\setminus\gamma$ has exactly two connected components.
    One of these, known as the \emph{interior}, is bounded, and the other, known as the \emph{exterior}, is unbounded.
\end{thm}

\begin{proof}
    The proof is far beyond the level of this course.
\end{proof}

\begin{defn}[Commonly used paths]
    Let $a,b\in\comps,r>0$.
    \begin{enumerate}[(i)]
        \item We write $\gamma(a,r)$ to mean the circle centre $a$, radius $r$, with a \emph{positive orientation}.
        That is, we traverse it anticlockwise.
        We parametrise it by
        \[
            z=a+re^{i\theta},\quad
            \theta\in[0,2\pi]
        \]

        We also define $\gamma^+(a,r)$ and $\gamma_-(a,r)$ to be the upper and lower semicircles, respectively.
        \item We write $[a,b]$ to mean the straight path joining $a$ and $b$.
        We parametrise is by
        \[
            z=a+t(b-a),\quad
            t\in[0,1]
        \]
    \end{enumerate}
\end{defn}

\begin{defn}[Reparametrisation]
    Given a path $\gamma:[a,b]\to\comps$, a \emph{reparametrisation} of $\gamma$ is a map $\Gamma:[c,d]\to\comps$ such that
    \[
        \Gamma=
        \gamma\circ\phi
    \]
    where $\phi:[c,d]\to[a,b]$ is a bijection with positive continuous derivative.
    That is, $\phi$ can be thought of as a change of coordinates, taking a points $\Gamma$ coordinate to its relative $\gamma$ coordinate.
\end{defn}

\begin{defn}[Length of a path]
    Given a path $\gamma:[a,b]\to\comps$, we define the \emph{length} of $\gamma$ as
    \begin{equation}
        \mathcal{L}(\gamma):=
        \int_a^b|\gamma'(t)|\diff t
    \end{equation}
\end{defn}

\begin{prop}[Length of a path under reparametrisation]
    The length of a path is invariant under reparametrisation
\end{prop}

\begin{proof}
    Let $\gamma_1:[a,b]\to\comps$ and $\gamma_2:[c,d]\to\comps$ be two parametrisations of the same path $\gamma$.
    Let $\alpha:[a,b]\to[c,d]$ be the change of coordinates map, so that $\gamma_2(\alpha(t))=\gamma_1(t)$.
    As $\alpha(a)=c$ and $\alpha(b)=d$, then
    \begin{align*}
        \int_a^b|\gamma_1'(t)|\diff t
        &=
        \int_a^b|\gamma_2'(\alpha(t))\alpha'(t)|\diff t\\
        &=
        \int_a^b|\gamma_2'(\alpha(t))|\alpha'(t)\diff t\\
        &=
        \int_c^d|\gamma_2'(u)|\diff u
    \end{align*}
\end{proof}

\begin{defn}[Path integrals]
    Let $f:U\to\comps$ be a continuous function defined on a domain $U$, and let $\gamma[a,b]\to U$ be a path in $U$.
    We define the \emph{path integral}
    \begin{equation}
        \int_{\gamma} f(z)\diff z:=
        \int_a^b f(\gamma(t))\gamma'(t)\diff t
    \end{equation}

    Note that the definition of complex integrals is exactly what one might expect.
    That is,
    \[
        \int_a^b (x(t)+iy(t))\diff t:=
        \int_a^b x(t)\diff t + i\int_a^b y(t)\diff t
    \]

    It is easy to check that integration remains linear when defined thusly.
\end{defn}

\begin{defn}[Join of paths]
    Given two paths, $\gamma_1:[a,b]\to\comps$ and $\gamma_2:[c,d]\to\comps$, with $\gamma_1(b)=\gamma_2(c)$, we define their \emph{join}, $\gamma_1\cup\gamma_2:[a,b+d-c]\to\comps$, by
    \begin{equation}
        (\gamma_1\cup\gamma_2)(t)=
        \left\{
        \begin{array}{lr}
            \gamma_1(t) & \text{for }t\in[a,b]\\
            \gamma_2(t-b+c) & \text{for }t\in[b,b+d-c]
        \end{array}
        \right.
    \end{equation}

    That is, the join of two paths traces out the first path, then the second.
\end{defn}

\begin{prop}[Joined path integral]
    For any continuous $f$ defined on $\gamma_1$ and $\gamma_2$,
    \[
        \int_{\gamma_1\cup\gamma_2} f(z)\diff z=
        \int_{\gamma_1} f(z)\diff z + \int_{\gamma_2} f(z)\diff z
    \]
\end{prop}

\begin{defn}[Reverse path]
    Given a path $\gamma:[a,b]\to\comps$ we can define the \emph{reverse path} $-\gamma:[a,b]\to\comps$ by
    \begin{equation}
        -\gamma(t)=
        \gamma(b+a-t)
    \end{equation}
\end{defn}

\begin{prop}[Reverse path integral]
    For any continuous $f$ defined on $\gamma$,
    \[
        \int_{-\gamma} f(z)\diff z=
        -\int_{\gamma} f(z)\diff z
    \]
\end{prop}

\begin{prop}[Path integrals under reparametrisation]
    The path integral $\int_{\gamma} f(z)\diff z$ is invariant under reparametrisation.
\end{prop}

\begin{proof}
    Let $\gamma_1:[a,b]\to\comps$ and $\gamma_2:[c,d]\to\comps$ be two parametrisations of the same path $\gamma$.
    Let $\alpha:[a,b]\to[c,d]$ be the change of coordinates map, so that $\gamma_2(\alpha(t))=\gamma_1(t)$.
    As $\alpha(a)=c$ and $\alpha(b)=d$, then, by the chain rule,
    \[
        \int_c^d f(\gamma_2(t))\gamma_2'(t)\diff t=
        \int_a^b f(\gamma_2(\alpha(t)))\gamma_2'(\alpha(t))\alpha'(t)\diff t=
        \int_a^b f(\gamma_1(t))\gamma_1'(t)\diff t
    \]
\end{proof}

\begin{prop}[Integration of centred power series terms]
    Let $a\in\comps,z\in\mathbb{Z},r>0$.
    Then
    \[
        \int_{\gamma(a,r)} (z-a)^k\diff z=
        \left\{
        \begin{array}{lr}
            2\pi i & \text{for }k=-1\\
            0 & \text{otherwise}
        \end{array}
        \right.
    \]
\end{prop}

\begin{proof}
    Let $z=a+re^{i\theta}$ with $\theta\in[0,2\pi]$.
    First assume that $k\neq-1$.
    Then
    \begin{align*}
        \int_{\gamma(a,r)} (z-a)^k\diff z
        &=
        \int_0^{2\pi} \left(re^{i\theta}\right)^k\left(ire^{i\theta}\diff\theta\right)\\
        &=
        ir^{k+1}\int_0^{2\pi} e^{i(k+1)\theta}\diff\theta\\
        &=
        ir^{k+1}\int_0^{2\pi} [\cos(k+1)\theta+i\sin(k+1)\theta]\diff\theta\\
        &=0
    \end{align*}

    Then, for $k=-1$,
    \[
        \int_{\gamma(a,r)} (z-a)^{-1}\diff z=
        ir^0\int_0^{2\pi}\diff\theta=
        2\pi i
    \]
\end{proof}

\begin{thm}[Fundamental Theorem of Calculus for path integrals]
    Let $f$ be a holomorphic function on domain $U$, and let $\gamma$ be a path in $U$ from $p$ to $q$.
    Then
    \begin{equation}
        \int_{\gamma} f'(z)\diff z=
        f(q)-f(p)
    \end{equation}
\end{thm}

\begin{proof}
    Let $\gamma:[a,b]\to U$ be a parametrisation of $\gamma$ with $\gamma(a)=p$ and $\gamma(b)=q$.
    Then
    \begin{align*}
        \int_{\gamma} f'(z)\diff z
        &=
        \int_a^b f'(\gamma(t))\gamma'(t)\diff t\\
        &=
        \int_a^b \frac{\mathrm{d}}{\mathrm{d}t} f(\gamma(t))\diff t\\
        &=
        [f(\gamma(t))]_{t=a}^{t=b}\\
        &=
        f(\gamma(b))-f(\gamma(a))\\
        &=
        f(q)-f(p)
    \end{align*}

    \emph{N.B. The application of the usual Fundamental Theorem of Calculus requires the derivative to be continuous, but as we will later prove, holomorphic functions have continuous derivatives.}
\end{proof}

\begin{cor}[Zero derivatives]
    If $f'=0$ on a domain $U$ then $f$ is constant.
\end{cor}

\begin{thm}[Estimation Theorem]
    Let $U$ be a domain, $\gamma:[a,b]\to U$ a path in $U$, and $f:U\to\comps$ be continuous.
    Then
    \begin{equation}
        \left|\int_{\gamma} f(z)\diff z\right|\leq
        \int_a^b |f(\gamma(t))\gamma'(t)|\diff t
    \end{equation}

    In particular, if $|f(z)|\leq M$ on $\gamma$, then
    \begin{equation}
        \left|\int_{\gamma} f(z)\diff z\right|\leq
        M\mathcal{L}(\gamma)
    \end{equation}
\end{thm}

\begin{proof}
    The proof that
    \[
        \left|\int_a^b z(t)\diff z\right|\leq
        \int_a^b |z(t)|\diff t
    \]
    for a complex integrand follows in a similar fashion to that for real integrands (being essentially a continuous version of the triangle inequality).
    We then have that
    \[
        \left|\int_{\gamma} f(z)\diff z\right|=
        \left|\int_a^b f(\gamma(t))\gamma'(t)\diff t\right|\leq
        \int_a^b |f(\gamma(t))\gamma'(t)|\diff t
    \]

    Further, if $|f(z)|\leq M$ on $\gamma$, then
    \[
        \left|\int_{\gamma}\diff z\right|\leq
        \int_a^b M|\gamma'(t)|\diff t=
        M\mathcal{L}(\gamma)
    \]
\end{proof}

\begin{prop}[Uniform convergence]
    Suppose that the functions $f_n(z)$ converge uniformly to $f(z)$ on the path $\gamma$.
    Then
    \begin{equation}
        \lim_{n\to\infty}\int_{\gamma} f_n(z)\diff z=
        \int_{\gamma} f(z)\diff z
    \end{equation}
\end{prop}

\begin{proof}
    Let $\eps>0$.
    There exists $N$ such that, for $n\geq N$ and $z\in\gamma$,
    \[
        |f_n(z)-f(z)|<
        \frac{\eps}{\mathcal{L}(\gamma)}
    \]
    Hence, for $n\geq N$, we have that
    \[
        \left|\left(\int_{\gamma} f_n(z)\diff z\right)-\left(\int_{\gamma} f(z)\diff z\right)\right|=
        \left|\int_{\gamma}(f_n(z)-f(z))\diff z\right|\leq
        \mathcal{L}(\gamma)\frac{\eps}{\mathcal{L}(\gamma)}=
        \eps
    \]
\end{proof}

\begin{thm}[Cauchy's Theorem, or, more correctly, The Cauchy-Goursat Theorem]
    Let $f(z)$ be holomorphic inside and on a closed path $\gamma$.
    Then
    \begin{equation}
        \int_{\gamma} f(z)\diff z
        =0
    \end{equation}
\end{thm}

\begin{proof}
    Once more, the proof lies well outside the level of this course.
    We will, however, prove the result for less general cases.
\end{proof}

\begin{thm}[Cauchy's Theorem for a triangle]
    Let $f$ be holomorphic in a domain which includes a closed triangular region $T$.
    Let $\Delta$ denote the boundary of $T$, positively oriented.
    Then
    \begin{equation}
        \int_{\Delta} f(z)\diff z
        =0
    \end{equation}
\end{thm}

\begin{proof}
    \emph{First, we create nested triangles and consider their intersection}:
    If we join the three midpoints of the sides of $\Delta$ to make four new triangular paths, A, B, C, and D, and orient them all positively, then we see that
    \[
        \int_{\Delta} f(z)\diff z=
        \int_A f(z)\diff z + \int_B f(z)\diff z + \int_C f(z)\diff z + \int_D f(z)\diff z
    \]
    as the contributions from the edges inside $\Delta$ cancel each other out with their opposite orientations.
    But one of these four integrals, over A, B, C, and D, must have the maximal modulus; call it $\Delta_1$.
    Then, by the triangle inequality,
    \[
        \left|\int_{\Delta} f(z)\diff z\right|\leq
        4\left|\int_{\Delta_1} f(z)\diff z\right|
    \]

    We can do this repeatedly, creating triangles $\Delta_1,\Delta_2,\ldots$ such that, for any $n$,
    \[
        \left|\int_{\Delta} f(z)\diff z\right|\leq
        4^n\left|\int_{\Delta_n} f(z)\diff z\right|        
    \]

    As the closed triangular regions (boundaries and interiors) are nested compact subsets then their intersection is non empty.
    \emph{(We prove this in a lemma at the end of this proof).}
    This intersection, however, cannot contain more than one point, as the triangles lie in discs whose radii tend to $0$.
    Let $\zeta$ denote the single complex number in the intersection.
    That is,
    \[
        \zeta=
        \bigcap_{n=1}^{\infty}\Delta_n
    \]

    \emph{Next, we apply the fact that $f$ is holomorphic, and therefore differentiable at $\zeta$}:
    As $f'(\zeta)$ exists, then, for any $\eps>0$, there exists some $\delta>0$ such that
    \[
        |z-\zeta|<\delta\implies
        \left|\frac{f(z)-f(\zeta)}{z-\zeta}-f'(\zeta)\right|<\eps
    \]
    or, equivalently, that, for some $w$ with $|w|<\eps$,
    \[
        f(z)=
        f(\zeta)+(z-\zeta)f'(\zeta)+(z-\zeta)w
    \]

    Now choose $n$ sufficiently large that $\Delta_n\subseteq D(\zeta,\delta)$, then
    \[
        \int_{\Delta_n} f(z)\diff z=
        f(\zeta)\int_{\Delta_n} \diff z+
        f'(\zeta)\int_{\Delta_n} (z-\zeta)\diff z+
        \int_{\Delta_n} (z-\zeta)w(z)\diff z
    \]

    By the Fundamental Theorem of Calculus, the first two integrals are zero, and thus
    \[
        \int_{\Delta_n} f(z)\diff z=
        \int_{\Delta_n} (z-\zeta)w(z)\diff z
    \]

    From the way we constructed the triangles, note that
    \[
        \mathcal{L}(\Delta_n)=
        \frac{1}{2}\mathcal{L}(\Delta_{n-1})=
        \frac{1}{2^n}\mathcal{L}(\Delta)
    \]
    and also, for any $z\in\Delta_n$, $|z-\zeta|<\mathcal{L}(\Delta_n)$.
    So, by the Estimation Theorem,
    \begin{align*}
        \left|\int_{\Delta_n} f(z)\diff z\right|
        &=
        \left|\int_{\Delta_n} (z-\zeta)w(z)\diff z \right|\\
        &\leq
        \mathcal{L}(\Delta_n)\mathcal{L}(\Delta_n)\eps\\
        &=
        \frac{\eps}{4^n}\mathcal{L}(\Delta)^2
    \end{align*}

    Finally,
    \[
        \left|\int_{\Delta} f(z)\diff z\right|\leq
        4^n\left|\int_{\Delta_n} f(z)\diff z\right|\leq
        \eps\mathcal{L}(\Delta)^2
    \]
    which, as $\eps>0$ was arbitrary, concludes the proof.
\end{proof}

\begin{lem}[Intersection of decreasing sequence]
    Let $M$ be a compact metric space and $C_n$ be a descreasing sequence of closed, non-empty subsets.
    Then
    \[
        \bigcap C_n\neq\varnothing
    \]
\end{lem}

\begin{proof}
    Let $U_n=M\setminus C_n$, so that the $U_n$ form an increasing sequence of open subsets.
    Suppose, for a contradiction, that $\bigcap C_n=\varnothing$.
    Then, by De Morgan's Law,
    \[
        \bigcup U_n=
        \bigcup (M\setminus C_n)=
        M\setminus\left(\bigcap C_n\right)=
        M
    \]
    and so the $U_n$ form an open cover for $M$.
    By compactness there are $n_1<n_2<\ldots<n_k$ such that
    \[
        U_{n_k}=
        U_{n_1}\cup\ldots\cup U_{n_k}=
        M
    \]
    but then $C_{n_k}=\varnothing$, which is a contradiction.
\end{proof}

\begin{defn}[Convex domains]
    A domain $U$ is \emph{convex} if
    \[
        a,b\in U\implies
        [a,b]\in U
    \]
\end{defn}

\begin{thm}[Antiderivative Theorem on a convex domain]
    Let $f$ be holomorphic on a convex domain $U$.
    Then there exists a holomorphic function $F$ on $U$ such that $F'(z)=f(z)$.
\end{thm}

\begin{proof}
    Fix $a\in U$, and let $[a,z]$ denote the oriented line segment from $a$ to $z$.
    Define
    \[
        F(z)=
        \int_{[a,z]} f(w)\diff w
    \]

    Let $\eps>0$ be such that $D(z,\eps)\subseteq U$, and take $h$ with $|h|<\eps$.
    By Cauchy's Theorem for triangles,
    \[
        \int_{[a,z]} f(w)\diff w+\int_{[z,z+h]} f(w)\diff w+\int_{[z+h,a]} f(w)\diff w=
        0
    \]
    which may be rearranged to
    \[
        F(z+h)-F(z)=
        \int_{[z,z+h]} f(w)\diff w
    \]

    So, as $\int_{[z,z+h]}k\diff w=kh$ for any constant function $k(w)$,
    \[
        \left|\frac{F(z+h)-F(z)}{h}-f(z)\right|=
        \left|\left(\frac{1}{h}\int_{[z,z+h]}f(w)\diff w\right)-f(z)\right|=
        \left|\frac{1}{h}\int_{[z,z+h]}[f(w)-f(z)]\diff w\right|
    \]

    Finally, by the Estimation Theorem, we have that
    \begin{align*}
        \left|\frac{F(z+h)-F(z)}{h}-f(z)\right|
        &=
        \left|\frac{1}{h}\int_{[z,z+h]}[f(w)-f(z)]\diff w\right|\\
        &\leq
        \frac{1}{|h|}|h|\sup_{w\in[z,z+h]}|f(w)-f(z)|\\
        &=
        \sup_{w\in[z,z+h]}|f(w)-f(z)|
    \end{align*}
    which tends to $0$ as $h\to0$ by the continuity of $f$ at $z$.
    Thus $F'(z)=f(z)$, as required.
\end{proof}

\begin{cor}[Cauchy's Theorem for a convex domain]
    Let $f$ be holomorphic on a convex domain $U$.
    Then for any closed path $\gamma$ in $U$, we have that
    \begin{equation}
        \int_{\gamma} f(z)\diff z=0
    \end{equation}
\end{cor}

\begin{proof}
    This follows from the Antiderivative Theorem and the Fundamental Theorem of Calculus.
\end{proof}

\begin{defn}[Homotopy]
    Let $\gamma_1:[0,1]\to U$ and $\gamma_2:[0,1]\to U$ be two closed paths in a domain $U$.
    We say that $\gamma_1$ and $\gamma_2$ are \emph{homotopic} if there is a continuous function $H:[0,1]^2\to\comps$ such that, for each $u\in[0,1]$, $H(_,u):[0,1]\to U$ is a closed path in $U$, and, for all $t\in[0,1]$,
    \[
        H(t,0)=\gamma_1(t),\quad
        H(t,1)=\gamma_2(t)
    \]

    We might think of the second variable $u$ as a time parameter.
    The homotopy $H$ can then be viewed as a continuous deformation from $\gamma_1$ to $\gamma_2$
\end{defn}

\begin{thm}[Deformation Theorem]
    Let $f$ be holomorphic on a domain $U$, and let $\gamma_1,\gamma_2$ be homotopic closed paths in $U$.
    Then
    \begin{equation}
        \int_{\gamma_1} f(z)\diff z=
        \int_{\gamma_2} f(z)\diff z
    \end{equation}
\end{thm}

\begin{proof}
    The full rigorous proof lies outside this course, but uses an appropriate version of Cauchy's Theorem on $\int_{\gamma_1\cup\gamma_2}f(z) \diff z$.
\end{proof}

\begin{defn}[Simple connectedness]
    A domain $U$ is \emph{simply connected} if every closed curve in $U$ is homotopic to a single point in $U$.
\end{defn}

\begin{cor}[Cauchy's Theorem on a simply-connected domain]
    Let $f$ be holomorphic on a simply-connected domain $U$, and let $\gamma$ be a closed path in $U$.
    Then
    \begin{equation}
        \int_{\gamma} f(z)\diff z=0
    \end{equation}
\end{cor}

\begin{proof}
    $\gamma$ is homotopic to a constant path, $\Gamma$, and so, by the Deformation Theorem,
    \[
        \int_{\gamma} f(z)\diff z=
        \int_{\Gamma} f(z)\diff z=
        0
    \]
\end{proof}

\begin{cor}[Path independence]
    Let $f$ be holomorphic on a simply-connected domain $U$.
    Let $a,b\in U$, and $\gamma_1,\gamma_2$ be two paths from $a$ to $b$.
    Then
    \[
        \int_{\gamma_1} f(z)\diff z=
        \int_{\gamma_2} f(z)\diff z
    \]
\end{cor}

\begin{proof}
    Apply Cauchy's Theorem on a simply-connected domain to the path $\gamma_1\cup\gamma_2$.
\end{proof}

\begin{cor}[Antiderivative Theorem for a simply-connected domain]
    Let $f$ be holomorphic on a simply-connected domain $U$.
    Then there exists $F$ on $U$ such that $F'=f$.
\end{cor}

\begin{proof}
    Fix $a\in U$, and for $z\in U$ let $\gamma$ be a path from $a$ to $z$ in $U$.
    Define
    \[
        F(z)=
        \int_{\gamma} f(w)\diff w
    \]

    By the previous corollary about path independence, $F$ is independent of the choice of path, and is thus a function of $z$ alone.
    By an argument identical to that for the Antiderivative Theorem for a convex domain, it follows that $F'(z)=f(z)$.
\end{proof}

\begin{cor}[Logarithm for a simply-connected domain]
    Let $U$ be a simply-connected domain not containing $0$.
    Then there exists a holomorphic function $l(z)$ on $U$ such that
    \[
        \exp(l(z))=z
    \]
\end{cor}

\begin{proof}
    Let $a\in U$ and $c\in[\log a]$, so that $\exp c=a$.
    For $z\in U$ let $\gamma$ be a path from $a$ to $z$.
    Define
    \[
        l(z)=
        c+\int_{\gamma}\frac{\mathrm{d}w}{w}
    \]

    By the Antideriative Theorem for simply-connected domains, $l'(z)=z^{-1}$.
    
    Further, define
    \[
        G(z)=
        \frac{\exp(l(z))}{z}
    \]
    Then, by the quotient rule,
    \[
        G'(z)=
        \frac{zl'(z)\exp(l(z))-\exp(l(z))}{z^2}=
        \frac{\exp(l(z))-\exp(l(z))}{z^2}
        =0
    \]
    and so $G$ is constant on $U$ by connectedness.
    At $z=a$ we see that
    \[
        G(a)=
        \frac{\exp(l(a))}{a}=
        \frac{\exp c}{a}=
        \frac{a}{a}
        =1
    \]
    and hence $\exp(l(z))=z$, as required.
\end{proof}


\section{Consequences of Cauchy's Theorem}

\begin{defn}[Orientation]
    Let $\gamma:[0,1]\to\comps$ be a simple path in $\comps$.
    Then $\gamma$ is a homeomorphism onto its image.
    If $\Gamma:[0,1]\to\comps$ is also a homeomorphism onto the image of $\gamma$ then $\gamma\circ\Gamma^{-1}$ is a homeomorphism of $[0,1]$.
    This means that $\gamma\circ\Gamma^{-1}$ is strictly monotone, and so either increasing or decreasing.

    We say that $\gamma$ and $\Gamma$ have the same \emph{orientation} if $\gamma\circ\Gamma^{-1}$ is increasing, and have \emph{reverse orientations} if $\gamma\circ\Gamma^{-1}$ is decreasing.

    If they have the same orientation then
    \[
        \int_{\gamma} f(z)\diff z=
        \int_{\Gamma} f(z)\diff z
    \]

    If they have reverse orientations then
    \[
        \int_{\gamma} f(z)\diff z=
        -\int_{\Gamma} f(z)\diff z        
    \]
\end{defn}

\begin{prop}
    Let $\gamma$ be a simple closed path in $\comps$.
    Let $a$ be in the interior of $\gamma$ (which exists, and is connected, by the Jordan Curve Theorem), and $r>0$ such that $\gamma(a,r)$ is also in the interior of $\gamma$.
    Then $\gamma$ is either homotopic to $\gamma(a,r)$ or $-\gamma(a,r)$
    So, by the Deformation Theorem, if $\gamma$ is homotopic to $\pm\gamma(a,r)$, then
    \[
        \int_{\gamma}\frac{\mathrm{d}z}{z-a}
        =\pm2\pi i
    \]
\end{prop}

\begin{defn}[Positive orientation]\label{positive-orientation}
    Let $\gamma$ be a simple closed curve and $a$ be a point in the interior of $\gamma$.
    Then $\gamma$ is positively oriented if
    \begin{equation}
        \int_{\gamma}\frac{\mathrm{d}z}{z-a}=
        2\pi i
    \end{equation}
\end{defn}

\begin{defn}[Winding number]
    Let $\gamma$ be a closed curve, and $a$ be a point not on $\gamma$.
    Then the \emph{winding number of $\gamma$ about $a$} is
    \begin{equation}
        \frac{1}{2\pi i}\int_{\gamma}\frac{\mathrm{d}z}{z-a}
    \end{equation}
\end{defn}

\begin{thm}[Cauchy's Integral Formula]
    Let $f$ be holomorphic on and inside a positively oriented, simple, closed curve $\gamma$, and $a$ be a point inside $\gamma$.
    Then
    \begin{equation}
        \frac{1}{2\pi i}\int_{\gamma}\frac{f(w)}{w-a}\diff w=
        f(a)
    \end{equation}
\end{thm}

\begin{proof}
    By the Deformation Theorem we know that
    \[
        \frac{1}{2\pi i}\int_{\gamma}\frac{f(w)}{w-a}\diff w=
        \frac{1}{2\pi i}\int_{\gamma(a,r)}\frac{f(w)}{w-a}\diff w
    \]
    for \emph{any} circle $\gamma(a,r)$ inside $\gamma$.

    By definition~\ref{positive-orientation} we know that
    \[
        \int_{\gamma(a,r)}\frac{\mathrm{d}w}{w-a}=
        2\pi i
    \]
    so that
    \[
        f(a)=
        \frac{1}{2\pi i}\int_{\gamma(a,r)}\frac{f(a)}{w-a}\diff w
    \]

    Hence
    \begin{align*}
        \left|\left(\frac{1}{2\pi i}\int_{\gamma(a,r)}\frac{f(w)}{w-a}\diff w\right)-f(a)\right|
        &=
        \left|\frac{1}{2\pi i}\int_{\gamma(a,r)}\frac{f(w)-f(a)}{w-a}\diff w\right|\\
        &=
        \frac{1}{2\pi}\left|\int_0^{2\pi}\frac{f(a+re^{i\theta})-f(a)}{re^{i\theta}}ire^{i\theta}\diff\theta\right|\\
        &=
        \frac{1}{2\pi}\left|\int_0^{2\pi}[f(a+re^{i\theta})-f(a)]\diff\theta\right|\\
        &\leq
        \frac{1}{2\pi}2\pi\sup_{\theta\in[0,2\pi]}|f(a+re^{i\theta})-f(a)|\\
        &=
        \sup_{\theta\in[0,2\pi]}|f(a+re^{i\theta})-f(a)|
    \end{align*}
    and, by the continuity of $f$ at $a$,
    \[
        \lim_{r\to0}\left[\sup_{\theta\in[0,2\pi]}|f(a+re^{i\theta})-f(a)|\right]
        =0
    \]
\end{proof}

\begin{thm}[Taylor's Theorem and Cauchy's Formula for Derivatives]
    Let $a\in\comps,\eps>0$, and $f:D(a,\eps)\to\comps$ be a holomorphic function.
    Then there exist unique $c_n\in\comps$ such that, for $z\in D(a,\eps)$,
    \begin{equation}
        f(z)=
        \sum_{n=0}^{\infty} c_n(z-a)^n
    \end{equation}

    Further, for $0<r<\eps$,
    \begin{equation}\label{cauchys-derivative}
        c_n=
        \frac{f^{(n)}(a)}{n!}=
        \frac{1}{2\pi i}\int_{\gamma(a,r)}\frac{f(w)}{(w-a)^{n+1}}\diff w
    \end{equation}

    Equation~\ref{cauchys-derivative} is often referred to as \emph{Cauchy's Formula for Derivatives}.
    We refer to the $c_n$ as the \emph{Taylor coefficients}.
\end{thm}

\begin{proof}
    \begin{enumerate}[(i)]
        \item \emph{Existence}:
        Choose $r$ such that $|z-a|<r<\eps$.
        By Cauchy's Integral Formula,
        \begin{align*}
            f(z)
            &=
            \frac{1}{2\pi i}\int_{\gamma(a,r)}\frac{f(w)}{w-z}\diff w\\
            &=
            \frac{1}{2\pi i}\int_{\gamma(a,r)}\frac{f(w)}{(w-a)-(z-a)}\diff w\\
            &=
            \frac{1}{2\pi i}\int_{\gamma(a,r)}\frac{f(w)/(w-a)}{1-\left(\frac{z-a}{w-a}\right)}\diff w
        \end{align*}

        For $w\in\gamma(a,r)$ we have $|z-a|<|w-a|=r$, and so
        \[
            \left(1-\left(\frac{z-a}{w-a}\right)\right)^{-1}=
            \sum_{n=0}^{\infty}\left(\frac{z-a}{w-a}\right)^n
        \]
        and hence
        \[
            f(z)=
            \frac{1}{2\pi i}\int_{\gamma(a,r)}\left[\sum_{n=0}^{\infty}f(w)\frac{(z-a)^n}{(w-a)^{n+1}}\right]\diff w
        \]

        As $\gamma(a,r)$ is compact and $f(w)$ continuous, then there exists $M>0$ such that $|f(w)|<M$ on $\gamma(a,r)$.
        So, for $w\in\gamma(a,r)$,
        \begin{align*}
            \left|f(w)\frac{(z-a)^n}{(w-a)^{n+1}}\right|
            &<
            M\frac{|z-a|^n}{r^{n+1}}\\
            &=
            \frac{M}{r}\left(\frac{|z-a|}{r}\right)^n\\
            &=:
            M_n
        \end{align*}

        As $\sum M_n$ is a convergent geometric series, then, by the Weierstrass M-Test,
        \[
            \sum_{n=0}^{\infty}f(w)\frac{(z-a)^n}{(w-a)^{n+1}}
        \]
        converges uniformly, and we can interchange the sum and the integral to find that
        \begin{align*}
            f(z)
            &=
            \frac{1}{2\pi i}\int_{\gamma(a,r)}\left[\sum_{n=0}^{\infty}f(w)\frac{(z-a)^n}{(w-a)^{n+1}}\right]\diff w\\
            &=
            \sum_{n=0}^{\infty}\underbrace{\left[\frac{1}{2\pi i}\int_{\gamma(a,r)}\frac{f(w)\mathrm{d}w}{(w-a)^{n+1}}\right]}_{c_n}(z-a)^n
        \end{align*}
        \item \emph{Uniqueness}:
        To demonstrate uniqueness we make use of the fact that, within the disc of convergence, the term-by-term derivative of a power series converges to the derivative of the function.
        Hence, if Taylor's Theorem holds, then
        \[
            f^{(k)}(z)=
            \sum_{n=0}^{\infty}\frac{n!c_n}{(n-k)!}(z-a)^{n-k}
        \]
        and, in particular, $c_k=\frac{f^{(k)}(a)}{k!}$, which is unique.
    \end{enumerate}
\end{proof}

\begin{cor}[Holomorphic functions have holomorphic derivatives]
    Let $f$ be holomorphic on a domain $U$.
    Then $f^{(n)}$ exists on $U$ and is holomorphic for all $n\geq0$.
\end{cor}

\begin{proof}
    Let $a\in U$ and $\eps>0$ such that $D(a,\eps)\subseteq U$.
    By Taylor's Theorem, we know that $f(z)=\sum_{n=0}^{\infty}a_nz^n$.
    From first year analysis we know that a power series defines a differentiable function on its disc of convergence, and that term-by-term differentiation is valid.
    Thus $f'(z)=\sum_{n=0}^{\infty}na_nz^{n-1}$.
    By induction, $f$ has derivatives of all orders.
\end{proof}

\begin{defn}[Zeros of a function]
    Let $f$ be holomorphic on some domain $U$, and let $a\in U$.
    Let the Taylor series of $f$ at $a$ be
    \[
        f(z)=
        \sum_{n=0}^{\infty}c_n(z-a)^n
    \]
    Then $a$ is a \emph{zero} of $f$ if $f(a)=c_0=0$, and the \emph{order} of the zero is $N$, where $N$ is the least $n$ such that $c_n\neq0$
\end{defn}

\begin{thm}[Liouville's Theorem]
    Let $f:\comps\to\comps$ be a holomorphic function which is bounded.
    Then $f$ is constant.
\end{thm}

\begin{proof}
    As $f$ is bounded then there exists $M$ such that $|f(z)|<M$ for all $z\in\comps$.
    By Taylor's Theorem we have that $f(z)=\sum_{n=0}^{\infty}c_nz^n$ for all $z\in\comps$.
    So, for $n\geq1$ and $r>0$, we have that
    \begin{align*}
        |c_n|
        &=
        \left|\frac{1}{2\pi i}\int_{\gamma(0,r)}\frac{f(w)}{w^{n+1}}\diff w\right|\\
        &\leq
        \frac{1}{2\pi}2\pi r\sup_{|w|=r}\left|\frac{f(w)}{w^{n+1}}\right|\\
        &\leq
        r\frac{M}{r^{n+1}}\\
        &=
        \frac{M}{r^n}\to0\quad\text{as }r\to\infty
    \end{align*}
    Hence $c_n=0$ for $n\geq1$ and $f(z)=c_0$ is constant.
\end{proof}

\begin{cor}[Density of holomorphic functions]
    Let $f$ be holomorphic on $\comps$ and non constant.
    Then $f(\comps)$ is dense in $\comps$.
    That is, $\overline{f(\comps)}=\comps$.
\end{cor}

\begin{proof}
    Take $a\in\comps$ and $\delta>0$, and suppose, for a contradiction, that $D(a,\delta)\subseteq\comps\setminus f(\comps)$.
    Then, for all $z\in\comps$,
    \[
        |f(z)-a|\geq\delta
    \]
    and so
    \[
        \frac{1}{|f(z)-a|}\leq\frac{1}{\delta}
    \]

    Hence $(f(z)-a)^{-1}$ is a bounded holomorphic function on $\comps$, and by Liouville's Theorem, constant.
    Thus $f(z)$ is constant, which is a contradiction.
\end{proof}

\begin{thm}[Fundamental Theorem of Algebra]
    Let $p$ be a non-constant polynomial with complex coefficients.
    Then there exists $\alpha\in\comps$ such that $p(\alpha)=0$.
\end{thm}

\begin{proof}
    Say that $p(z)=a_nz^n+\ldots+a_0$, where $n\geq1,a_i\in\comps$, and $a_n\neq0$.
    As $\lim_{z\to\infty}p(z)/z^n=a_n$, then there exists $R>0$ such that, for $z$ with $|z|>R$,
    \[
        \left|\frac{p(z)}{z^n}\right|>\frac{|a_n|}{2}
    \]

    Suppose, for a contradiction, that $p$ has no roots, so that $1/p$ is holomorphic.
    Then, by Cauchy's Integral Formula, with $r>R$, we have that
    \begin{align*}
        0
        &\neq
        \left|\frac{1}{p(0)}\right|\\
        &=
        \left|\frac{1}{2\pi i}\int_{\gamma(0,r)}\frac{1/f(w)}{w}\diff w\right|\\
        &\leq
        \frac{1}{2\pi}2\pi r\sum_{|w|=r}\left|\frac{1}{wp(w)}\right|\\
        &\leq
        \frac{1}{2\pi}2\pi r\frac{2}{|a_n|r^{n+1}}\\
        &=
        \frac{2}{|a_n|r^n}\to0\quad\text{as }r\to\infty
    \end{align*}
    which is a contradiction.
\end{proof}

\begin{thm}[Morera's Theorem]
    Let $f:U\to\comps$ be a continuous function on a domain $U$ such that
    \[
        \int_{\gamma} f(z)\diff z=0
    \]
    for any closed path $\gamma$ in $U$.
    Then $f$ is holomorphic.
\end{thm}

\begin{proof}
    Let $z_0\in U$.
    As $U$ is open and connected then it is path connected, and so for any $z\in U$ there is a path $\gamma(z)$ connecting $z_0$ to $z$.
    Define
    \[
        F(z)=
        \int_{\gamma(z)} f(w)\diff w
    \]
    Note that if $\gamma_1$ and $\gamma_2$ are two such paths, then $\gamma_1\cup(-\gamma_2)$ is a closed path, and by the hypothesis and properties of joins of paths, we have that $F(z)$ is the same for either choice of path.
    That is, $F(z)$ is well defined.

    Now choose $r>0$ such that $D(z_0,r)\subseteq U$ and choose $h\in D(z_0,r)$.
    Then
    \[
        F(z_0+h)=
        F(z_0)+\int_{[z_0,z_0+h]} f(w)\diff w
    \]

    Hence, by the Estimation Theorem, and the continuity of $f$ at $z_0$,
    \begin{align*}
        \left|\frac{F(z_0+h)-F(z_0)}{h}-f(z_0)\right|
        &=
        \left|\left(\frac{1}{h}\int_{[z_0,z_0+h]} f(w)\diff w\right)-f(z_0)\right|\\
        &=
        \left|\frac{1}{h}\int_{[z_0,z_0+h]} [f(w)-f(z_0)]\diff w\right|\\
        &\leq
        \frac{1}{|h|}|h|\sup_{w\in[z_0,z_0+h]}|f(w)-f(z_0)|\\
        &=
        \sup_{w\in[z_0,z_0+h]}|f(w)-f(z_0)|\to0\quad\text{as }h\to0
    \end{align*}

    Thus $F$ is holomorphic and $F'=f$, which is therefore also holomorphic.
\end{proof}

\begin{thm}[Identity Theorem]
    Let $f$ be holomorphic on a domain $U$.
    Then the following are equivalent:
    \begin{enumerate}[(i)]
        \item $f(z)=0$ for all $z\in U$
        \item The zero set $f^{-1}(0)$ has a limit point in $U$
        \item There exists $a\in U$ such that $f^{(k)}(a)=0$ for all $k\geq 0$
    \end{enumerate}
\end{thm}

\begin{proof}
    We shall prove that $(i)\implies(iii)\implies(ii)\implies(i)$.

    \begin{enumerate}
        \item[$(i)\implies(iii)$:]
        Choose $a$ to be any $z\in U$.
        \item[$(iii)\implies(ii)$:]
        As $U$ is open, there exists $\eps>0$ such that $D(a,\eps)\subseteq U$.
        By Taylor's Theorem, for $z\in D(a,\eps)$, we have
        \[
            f(z)=
            \sum_{k=0}^{\infty}\frac{f^{(k)}(a)}{k!}(z-a)^k=
            0
        \]

        Thus $D(a,\eps)\subseteq f^{-1}(0)$, and so $f^{-1}(0)$ has a limit point in $U$.
        \item[$(ii)\implies(i)$:]
        Let $a$ be a limit point of $f^{-1}(0)$ in $U$.
        Let $\eps>0$ be such that $D(a,\eps)\subseteq U$, and let
        \[
            f(z)=
            \sum_{k=0}^{\infty}c_k(z-a)^k
        \]
        be the Taylor expansion of $f$ centred about $a$.
        Suppose, for a contradiction, that not all $c_k$ are zero, and let $K$ be the smallest $k$ such that $c_k\neq0$.

        Define
        \[
            g(z)=
            \sum_{k=K}^{\infty}c_k(z-a)^{k-K}
        \]
        and then
        \[
            f(z)=
            (z-a)^Kg(z)
        \]

        Note that $g(z)$ is holomorphic on $D(a,\eps)$, and $g(a)=c_K\neq0$.
        By continuity there exists $\delta>0$ such that $g(z)\neq0$ on $D(a,\delta)$.
        Thus, in $D(a,\delta)$, we see that $f(z)=0$ holds only at $z=a$, contradicting the fact that $a$ is a limit point of $f^{-1}(0)$.
        Hence $c_k=0$ for all $k\geq0$, and so $f(z)=0$ on $D(a,\eps)$.

        Now let $S$ denote the set of all limits points of $f^{-1}(0)$.
        By assumption $S\neq\varnothing$.
        As $f$ is continuous, then $f^{-1}(0)$ is closed, and so $S\subseteq f^{-1}(0)$.
        By the previous part of the argument, if $a\in S$ then $D(a,\eps)\subseteq S$ for some $\eps>0$, and hence $S$ is open.

        Finally, if $z\in U\setminus S$, then $z$ is not a limit point of $f^{-1}(0)$ and so there exists $r>0$ such that $D(z,r)\subseteq U\setminus S$, and we see that $U\setminus S$ is open, and thus $S$ is closed.

        As $S$ is both open and closed, and $U$ is connected, then $U=S\subseteq f^{-1}(0)$.
    \end{enumerate}
\end{proof}

\begin{cor}
    Let $f$ and $g$ be holomorphic on a domain $U$.
    Then the following are equivalent:
    \begin{enumerate}[(i)]
        \item $f(z)=g(z)$ for all $z\in U$
        \item $f(z)=g(z)$ for all $z\in S$, where $S\subseteq U$ has a limit point in $U$
        \item There exists $a\in U$ such that $f^{(k)}(a)=g^{(k)}(a)$ for all $k\geq0$
    \end{enumerate}
\end{cor}

\begin{proof}
    Apply the identity theorem to $f-g$.
\end{proof}

\begin{prop}[Counting zeros]
    Let $f$ be holomorphic inside and on a postively-oriented closed path $\gamma$.
    Assume further that $f$ is non-zero on $\gamma$.
    Then
    \begin{equation}
        \text{number of zeros of }f\text{ in }\gamma\text{ (counting multiplicites)}=
        \frac{1}{2\pi i}\int_{\gamma}\frac{f'(z)}{f(z)}\diff z
    \end{equation}
\end{prop}

\begin{proof}
    Let the zeros of $f$ be $a_1,\ldots,a_k$, with multiplicities $m_1,\ldots,m_k$.
    Then $f'(z)/f(z)$ is holomorphic inside $\gamma$ except at the $a_i$.
    By Taylor's Theorem, we know that, in an open disc $D(a_i,r_i)$ around $a_i$, we can write
    \begin{align*}
        f(z)
        &=
        \sum_{k=m_i}^{\infty}c_k(z-a_i)^k\\
        &=
        (z-a_i)^{m_i}\underbrace{\sum_{k=m_i}^{\infty}c_k(z-a_i)^{k-m_i}}_{g(z)}\\
        &=
        (z-a_i)^{m_i}g(z)
    \end{align*}
    where $g(z)$ is holomorphic and $g(a_i)\neq0$.

    So in $D'(a_i,r_i)$ we have that
    \[
        \frac{f'(z)}{f(z)}=
        \frac{m_i(z-a_i)^{m_i-1}g(z)+(z-a_i)^{m_i}g'(z)}{(z-a_i)^{m_i}g(z)}=
        \frac{m_i}{z-a_i}+\frac{g'(z)}{g(z)}
    \]
    In particular,
    \[
        \frac{f'(z)}{f(z)}-\frac{m_i}{z-a_i}=
        \frac{g'(z)}{g(z)}
    \]
    is holomorphic in $D(a,r)$.
    So, similarly, we see that
    \[
        F(z)=
        \frac{f'(z)}{f(z)}-\sum_{i=1}^{k}\frac{m_i}{z-a_i}
    \]
    is holomorphic inside and on $\gamma$, having been suitably adjusted at each zero $a_i$.

    By Cauchy's Theorem,
    \begin{align*}
        0
        &=
        \int_{\gamma} F(z)\diff z\\
        &=
        \int_{\gamma} \frac{f'(z)}{f(z)}\diff z-\sum_{i=1}^{k}\left(\int_{\gamma}\frac{m_i}{z-a_i}\diff z\right)\\
        &=
        \int_{\gamma} \frac{f'(z)}{f(z)}\diff z-2\pi i\sum_{i=1}^k m_i
    \end{align*}
\end{proof}


\section{Laurent's Theorem}

\begin{thm}[Laurent's Theorem]
    Let $f$ be holomorphic on the annulus
    \[
        A=
        \{z\in\comps\colon R<|z-a|<S\}
    \]

    Then there exist unique $c_k$ ($k\in\mathbb{Z}$) such that, for $z\in A$,
    \[
        f(z)=
        \sum_{-\infty}^{\infty}c_k(z-a)^k
    \]
    where, for $R<r<S$,
    \[
        c_k=
        \frac{1}{2\pi i}\int_{\gamma(a,r)}\frac{f(w)}{(w-a)^{k+1}}\diff w
    \]
\end{thm}

\begin{proof}
    \begin{enumerate}[(i)]
        \item \emph{Existence}:
        Let $z\in A$ and choose $P,Q$ such that
        \[
            R<P<|z-a|<Q<S
        \]

        Let $\gamma_1$ be the path $\gamma^+(a,Q)\cup[a-Q,a-P]\cup(-\gamma^+(a,P))\cup[a+P,a+Q]$.
        Similarly let $\gamma_2$ be the path $\gamma^-(a,Q)\cup[a+Q,a+P]\cup(-\gamma^-(a,P))\cup[a-P,a-Q]$.

        By Cauchy's Integral Formula and Cauchy's Theorem, respectively, we know that
        \begin{align*}
            f(z)
            &=
            \frac{1}{2\pi i}\int_{\gamma_1}\frac{f(w)}{w-z}\diff w\\
            0
            &=
            \frac{1}{2\pi i}\int_{\gamma_2}\frac{f(w)}{w-z}\diff w
        \end{align*}

        As the path integrals along the internal line segments cancel out, we then have that
        \begin{align*}
            f(z)
            &=
            \frac{1}{2\pi i}\int_{\gamma_1}\frac{f(w)}{w-z}\diff w+\frac{1}{2\pi i}\int_{\gamma_2}\frac{f(w)}{w-z}\diff w\\
            &=
            \frac{1}{2\pi i}\int_{\gamma(a,Q)}\frac{f(w)}{w-z}\diff w-\frac{1}{2\pi i}\int_{\gamma(a,P)}\frac{f(w)}{w-z}\diff w
        \end{align*}

        For $w\in\gamma(a,Q)$, note that $|z-a|<|w-a|$, and for $w\in\gamma(a,P)$, note that $|z-a|>|w-a|$.
        Hence
        \begin{align*}
            f(z)
            &=
            \frac{1}{2\pi i}\int_{\gamma(a,Q)}\frac{f(w)/(w-a)}{1-\frac{z-a}{w-a}}\diff w+\frac{1}{2\pi i}\int_{\gamma(a,P)}\frac{f(w)/(z-a)}{1-\frac{w-a}{z-a}}\diff w\\
            &=
            \frac{1}{2\pi i}\int_{\gamma(a,Q)}\sum_{k=0}^{\infty}\frac{f(w)(z-a)^k}{(w-a)^{k+1}}\diff w+\frac{1}{2\pi i}\int_{\gamma(a,P)}\sum_{k=0}^{\infty}\frac{f(w)(w-a)^k}{(z-a)^{k+1}}\diff w
        \end{align*}

        Arguing as in the proof for Taylor's Theorem, with the Weierstrass M-Test, we may show that these sums converge uniformly.
        Hence we may change the order of integration and summation to obtain
        \begin{align*}
            f(z)
            &=
            \sum_{k=0}^{\infty}\left(\frac{1}{2\pi i}\int_{\gamma(a,Q)}\frac{f(w)}{(w-a)^{k+1}}\diff w\right)(z-a)^k\\
            &\quad+
            \sum_{k=0}^{\infty}\left(\frac{1}{2\pi i}\int_{\gamma(a,P)}f(w)(w-a)^k\diff w\right)(z-a)^{-k-1}\\
            &=
            \sum_{k=0}^{\infty}\left(\frac{1}{2\pi i}\int_{\gamma(a,r)}\frac{f(w)}{(w-a)^{k+1}}\diff w\right)(z-a)^k\\
            &\quad+
            \sum_{k=0}^{\infty}\left(\frac{1}{2\pi i}\int_{\gamma(a,r)}f(w)(w-a)^k\diff w\right)(z-a)^{-k-1}\\
            &=
            \sum_{k=-\infty}^{\infty}\left(\frac{1}{2\pi i}\int_{\gamma(a,r)}\frac{f(w)}{(w-a)^{k+1}}\diff w\right)(z-a)^k
        \end{align*}
        as required.
        \item \emph{Uniqueness}:
        Suppose now that, for $z\in A$,
        \[
            f(z)=
            \sum_{k=-\infty}^{\infty}d_k(z-a)^k
        \]

        For $R<r<S$, we have that
        \begin{align*}
            2\pi ic_n
            &=
            \int_{\gamma(a,r)}\frac{f(w)}{(w-a)^{n+1}}\diff w\\
            &=
            \int_{\gamma(a,r)}\sum_{k=-\infty}^{\infty}d_k(w-a)^{k-n-1}\diff w\\
            &=
            \left(\int_{\gamma(a,r)}\sum_{k=0}^{\infty}d_k(w-a)^{k-n-1}\diff w\right)\\
            &\quad+
            \left(\int_{\gamma(a,r)}\sum_{k=1}^{\infty}d_{-k}(w-a)^{-k-n-1}\diff w\right)
        \end{align*}

        As the seperate power series converge uniformly on $\gamma(a,r)$ then we may exchange the order of integration and summation, and find that
        \[
            2\pi ic_n=
            \sum_{k=-\infty}^{\infty}\left(\int_{\gamma(a,r)}\sum_{k=0}^{\infty}d_k(w-a)^{k-n-1}\diff w\right)=
            2\pi id_n
        \]
    \end{enumerate}
\end{proof}

\begin{rem}
    We almost never use the integral expression to find the Laurent coefficients.
    Instead, we tend to use standard power series for familiar functions to determinte the Laurent series.
\end{rem}

\begin{prop}[Equality of Laurent and Taylor series]
    If $f$ is holomorphic at $a$ then, by uniqueness, the Laurent series of $f$ centred at $a$ is simply the Taylor series of $f$ centred at $a$.
\end{prop}

\begin{defn}[Regular points, singularities, and isolated points]
    Let $f:U\to\comps$ be defined on a domain $U$.
    Then $a\in U$ is a(n)
    \begin{enumerate}[(i)]
        \item \emph{regular point} if $f$ is holomorphic at $a$
        \item \emph{singularity} if $f$ is not holomorphic at $a$, but $a$ is a limit point of regular points
        \item \emph{isolated singularity} if $a$ is a singularity, but $f$ is holomorphic on some $D'(a,r)\subseteq U$
    \end{enumerate}
\end{defn}

\begin{defn}[Residues, poles, and singularities]
    Let $a$ be an isolated singularity of $f$.
    By Laurent's Theorem, there exist unique $c_n$ such that, for $z\in D'(a,\eps)$,
    \[
        f(z)=
        \sum_{n=-\infty}^{\infty}c_n(z-a)^n
    \]

    Then
    \begin{enumerate}[(i)]
        \item The \emph{principal part of $f$ at $a$} is
        \[
            \sum_{n=-\infty}^{-1}c_n(z-a)^n
        \]
        \item The \emph{residue of $f$ at $a$} is $c_{-1}$
        \item $a$ is a \emph{removable singularity of $f$} if $c_n=0$ for all $n<0$
        \item $a$ is an \emph{essential singularity of $f$} if there are infinitely many negative $n$ such that $c_n\neq0$
        \item $a$ is said to be a \emph{pole of order $k$} if $c_{-k}\neq0$ and $c_n=0$ for all $n<-k$.
        Poles of order one are referred to as \emph{simple poles}.
    \end{enumerate}
\end{defn}

\begin{prop}
    Let $f,g:U\to\comps$ be holomorphic and $a\in U$.
    Suppose that $f$ has a zero of order $m$ at $a$ and $g$ has a zero of order $n$ at $a$.
    Then
    \[
        f/g\text{ has }
        \left\{
        \begin{array}{lr}
            \text{a pole of order }n-m\text{ at }a & \text{if }m<n\\
            \text{a removable singularity at }a & \text{if }m\geq n
        \end{array}
        \right.
    \]
\end{prop}

\begin{proof}
    Given the hypothesis, $f(z)=(z-a)^mF(z)$ and $g(z)=(z-a)^nG(z)$, where $F$ and $G$ are holomorphic on $U$, and $F(a)\neq0\neq G(a)$.
    Hence, for $z\neq a$,
    \[
        \frac{f(z)}{g(z)}=
        (z-a)^{m-n}\frac{F(z)}{G(z)}
    \]
    where $F/G$ is holomorphic about $a$ and non zero.
    This proves the first case.

    Now note that, if $m\geq n$, then there is still a singularity at $a$, as the function is undefined, but if this singularity were removed then $f/g$ would have a zero of order $m-n$ at $a$.
\end{proof}

\begin{prop}[Residues at simple poles]
    Suppose that $f(z)$ has a simple pole at $a$.
    Then
    \begin{equation}
        \res(f;a)=
        \lim_{z\to a}[(z-a)f(z)]
    \end{equation}

    Further, if $f(z)=g(z)/h(z)$, where $g$ and $h$ are holomorphic at $a$, and $h$ has a simple zero at $a$, then
    \begin{equation}
        \res(f;a)=
        \frac{g(a)}{h'(a)}
    \end{equation}

    Thus if $f(z)=g(z)/(z-a)$, then $\res(f;a)=g(a)$.

\end{prop}

\begin{proof}
    Let $R=\res(f;a)$.
    Then there exists a holomorphic function $h$ such that, on some $D'(a,\eps)$,
    \[
        f(z)=
        \frac{R}{z-a}+h(z)
    \]
    and so
    \[
        (z-a)f(z)=
        R+(z-a)h(z)\to R\quad\text{as }z\to a
    \]

    If $f(z)=g(z)/h(z)$ as given, then, by Taylor's Theorem,
    \begin{align*}
        g(z)
        &=
        g(a)+\mathcal{O}(z-a)\\
        h(z)
        &=
        h'(a)(z-a)+\mathcal{O}((z-a)^2)
    \end{align*}
    so that
    \begin{align*}
        \lim_{z\to a}[(z-a)f(z)]
        &=
        \lim_{z\to a}\left[\frac{(z-a)g(z)}{h(z)}\right]\\
        &=
        \lim_{z\to a}\left[\frac{(z-a)(g(a)+\mathcal{O}(z-a))}{h'(a)(z-a)+\mathcal{O}((z-a)^2)}\right]\\
        &=
        \lim_{z\to a}\left[\frac{g(a)+\mathcal{O}(z-a)}{h'(a)+\mathcal{O}(z-a)}\right]\\
        &=
        \frac{g(a)}{h'(a)}
    \end{align*}
\end{proof}

\begin{prop}[Residues at overt multiple poles]
    Suppose that
    \[
        f(z)=
        \frac{g(z)}{(z-a)^n}
    \]
    where $g$ is holomorphic at $a$ and $g(a)\neq0$.
    (That is, $f$ has an overt pole of order $n$ at $a$.)
    Then
    \begin{equation}
        \res(f;a)=
        \frac{g^{(n-1)}(a)}{(n-1)!}
    \end{equation}
\end{prop}

\begin{proof}
    By Taylor's Theorem, we can write, on some $D(a,\eps)$,
    \[
        g(z)=
        \sum_{k=0}^{\infty}c_k(z-a)^k
    \]
    where
    \[
        c_k=
        \frac{g^{(k)}(a)}{k!}
    \]

    So on $D'(a,\eps)$ we have that
    \[
        f(z)=
        \sum_{k=0}^{\infty}c_k(z-a)^{k-n}
    \]
    and we see that $\res(f;a)=c_{n-1}$, as required.
\end{proof}

\begin{rem}
    Note that, if a function is even about $a$, then its residue will be 0, simply because the Laurent expansion will only involve even powers of $z$.
\end{rem}


\section{Cauchy's Residue Theorem and its applications}

\begin{thm}[Cauchy's Residue Theorem]
    Let $f$ be holomorphic inside and on a simple, closed, positively-oriented path $\gamma$, except at points $a_1,\ldots,a_n$ inside $\gamma$.
    Then
    \begin{equation}
        \int_{\gamma} f(z)\diff z=
        2\pi i\sum_{k=1}^n\res(f;a_k)
    \end{equation}
\end{thm}

\begin{proof}
    On a disc $D(a_k,\eps_k)$ we can expand $f(z)$ as a Laurent expansion:
    \[
        f(z)=
        \underbrace{\sum_{n=0}^{\infty}c_n^{(k)}(z-a_k)^n}_{g_k(z)}+
        \underbrace{\sum_{n=-\infty}^{-1}c_n^{(k)}(z-a_k)^n}_{h_k(z)}=
        g_k(z)+h_k(z)
    \]
    so that $h_k(z)$ is the principal part of $f$ around $a_k$.
    Note that the sum defining $h_k(z)$ converges, except at $a_k$.
    Hence
    \[
        F(z)=
        f(z)-\sum_{k=1}^nh_k(z)
    \]
    is holomorphic in $\gamma$ except for removable singularities at $a_1,\ldots,a_n$.
    As we have seen, each $F(a_k)$ may be assigned a value so as to extend $F$ to a holomorphic function in $\gamma$.

    By Cauchy's Theorem and the Deformation Theorem, for $0<r_k<\eps_k$,
    \[
        \int_{\gamma} f(z)\diff z=
        \sum_{k=1}^n\int_{\gamma} h_k(z)\diff z=
        \sum_{k=1}^n\int_{\gamma(a_k,r_k)} h_k(z)\diff z
    \]
    As we saw in the proof of Laurent's Theorem, the sum defining $h_k(z)$ converges uniformly on $\gamma(a_k,r_k)$, and hence
    \begin{align*}
        \int_{\gamma(a_k,r_k)} h_k(z)\diff z
        &=
        \int_{\gamma(a_k,r_k)} \sum_{n=-\infty}^{-1}\left[c_n^{(k)}(z-a_k)^n\right]\diff z\\
        &=
        \sum_{n=-\infty}^{-1}c_n^{(k)}\left[\int_{\gamma(a_k,r_k)} (z-a_k)^n\diff z\right]\\
        &=
        2\pi ic_{-1}^{(k)}
    \end{align*}
    and the result follows.
\end{proof}

\begin{rem}[Standard contours]
    We often use the Residue Theorem to calculate real integrals.
    What follows is a list of the types of real integrals we often calculate with this theorem, as well as the contours we use to calculate them:
    \begin{enumerate}[(i)]
        \item \emph{Integrals from $0$ to $2\pi$}:
        We use the circle $\gamma(0,1)$, and then the substitution $z=e^{i\theta}$
        \item \emph{Integrals from $0$ or $-\infty$ to $\infty$}:
        We use the semicircle $\gamma^+(0,R)\cup[-R,R]$, with the aim of showing that the contribution from $\gamma^+(0,R)\to0$ as $R\to\infty$.
        If the integral is even then the integral on $(0,\infty)$ will be half that of the integral on $(-\infty,\infty)$.
        \item \emph{Infinite sums}:
        We use the square $\Gamma_N$, with vertices at $(N+\frac{1}{2})(\pm1\pm i)$, where $N$ is a positive integer, and aim to show that the integral tends to 0 as $N\to\infty$.
    \end{enumerate}
\end{rem}

\begin{prop}
    Suppose that the function $\phi(z)$ is holomorphic at $z=n\in\mathbb{Z}$, with $\phi(n)\neq0$.
    Then
    \begin{enumerate}[(i)]
        \item $\pi\phi(z)\cot\pi z$ has a simple pole at $n$ with residue $\phi(n)$
        \item $\pi\phi(z)\csc\pi z$ has a simple pole at $n$ with residue $(-1)^n\phi(n)$
    \end{enumerate}
\end{prop}

\begin{proof}
    Note that $\tan\pi z$ and $\sin\pi z$ have simple zeros at $z=n$, and hence $\pi\phi(z)\cot\pi z$ and $\pi\phi(z)\csc\pi z$ have simple poles there.
    So
    \[
        \res\left(\frac{\pi\phi(z)}{\tan\pi z};n\right)=
        \frac{\pi\phi(n)}{\pi\sec^2\pi n}=
        \phi(n),\qquad
        \res\left(\frac{\pi\phi(z)}{\sin\pi z};n\right)=
        \frac{\pi\phi(n)}{\pi\cos\pi n}=
        (-1)^n\phi(n)
    \]
\end{proof}

\begin{lem}[Jordan's Lemma]
    Let $\theta\in(0,\pi/2)$.
    Then
    \begin{equation}
        \frac{2}{\pi}<
        \frac{\sin\theta}{\theta}<
        1
    \end{equation}
\end{lem}

\begin{proof}
    Throughout the proof, let $\theta\in(0,\pi/2)$
    Let $f(\theta)=\sin\theta/\theta$.
    Note that
    \[
        f'(\theta)=
        \frac{\theta\cos\theta-\sin\theta}{\theta^2}
    \]

    Let $g(\theta)=\theta\cos\theta-\sin\theta$.
    Then
    \[
        g'(\theta)=
        -\theta\sin\theta+\cos\theta-\cos\theta=
        -\theta\sin\theta<
        0
    \]
    so that $g$ is decreasing.
    Further, as $g(0)=0$, then $g(\theta)<0$, and
    \[
        f'(\theta)=
        \frac{g(\theta)}{\theta^2}<
        0
    \]
    so that $f$ is decreasing.

    As $f(\theta)\to1$ as $\theta\to0$, we have that
    \[
        \frac{2}{\pi}<
        f(\theta)<
        1
    \]
\end{proof}

\begin{prop}[Indentation of contours]
    Let $f$ be holomorphic on $D'(a,\eps)$ with a simple pole at $a$.
    Let $0<r<\eps$, and let $\gamma_r$ be the positively oriented arc
    \[
        \gamma_r(\theta)=
        a+re^{i\theta},\qquad
        \theta\in(\alpha,\beta)
    \]
    Then
    \[
        \lim_{r\to0}\int_{\gamma_r} f(z)\diff z=
        i(\beta-\alpha)\res(f;a)
    \]
\end{prop}

\begin{proof}
    Let $R=\res(f;a)$.
    There is a holomorphic function $h$ on $D'(a,\eps)$ such that
    \[
        f(z)=
        \frac{R}{z-a}+h(z)
    \]

    Now
    \[
        \int_{\gamma_r}\frac{\mathrm{d}z}{z-a}=
        \int_{\alpha}^{\beta}\frac{ire^{i\theta}}{re^{i\theta}}\diff\theta=
        i\int_{\alpha}^{\beta}\diff\theta=
        i(\beta-\alpha)
    \]

    Further, $h$ is bounded on $\bar{D}(a,\eps/2)$ (as the former is continuous and the latter compact), and so $h$ is also bounded on every $\gamma_r$, with $r<\eps/2$, say by $M$.
    So, by the Estimation Theorem,
    \[
        \left|\int_{\gamma_r} h(z)\diff z\right|\leq
        M\mathcal{L}(\gamma_r)=
        Mr(\beta-\alpha)\to0\quad\text{as }r\to0
    \]

    Finally
    \begin{align*}
        \lim_{r\to0}\int_{\gamma_r} f(z)\diff z
        &=
        \lim_{r\to0}\int_{\gamma_r}\frac{R\diff z}{z-a} + \lim_{r\to0}\int_{\gamma_r} h(z)\diff z\\
        &=
        iR(\beta-\alpha)+0\\
        &=
        i(\beta-\alpha)R
    \end{align*}
    as required.
\end{proof}


\section{The Riemann Sphere and M\"obius transformations}

\begin{defn}[Stereographic projection]
    Let $S^2\subseteq\reals^3$ denote the unit sphere $x^2+y^2+z^2=1$, and let $N=(0,0,1)$ denote the `north pole' of $S^2$.
    Given a point $M\in S^2$, other than $N$, the line connecting $N$ and $M$ intersects the $xy$-plane at a point $P$.
    If we identify the $xy$-plane with $\comps$ in the natural way, then the \emph{stereographic projection} is the map
    \begin{equation}
        \pi:S^2\setminus\{N\}\to\comps,\quad
        \text{given by }M\mapsto P
    \end{equation}
\end{defn}

\begin{prop}[The stereographic map]
    The map $\pi$ is given by
    \begin{equation}
        \pi(a,b,c)=
        \frac{a+ib}{1-c}
    \end{equation}

    The inverse map is given by
    \begin{equation}
        \pi^{-1}(x+iy)=
        \frac{(2x,2y,x^2+y^2-1)}{x^2+y^2+1}
    \end{equation}
\end{prop}

\begin{proof}
    Say $M=(a,b,c)$.
    Then the line connecting $M$ and $N$ can be written parametrically as
    \[
        \vc{r}(t)=
        (0,0,1)+t(a,b,c-1)
    \]
    This intersects the $xy$-plane when $1+t(z-1)=0$.
    That is, when $t=(1-z)^{-1}$.
    Hence
    \[
        P=
        \vc{r}\left(\frac{1}{1-c}\right)=
        \left(\frac{a}{1-c},\frac{b}{1-c},0\right)
    \]
    which is naturally identified with
    \[
        \frac{a+ib}{1-c}\in\comps
    \]

    On the other hand, if $\pi(a,b,c)=x+iy$, then
    \[
        \frac{a+ib}{1-c}=x+iy
        \quad\text{and}\quad
        a^2+b^2+c^2=1
    \]
    so
    \begin{align*}
        x^2+y^2
        &=
        (x+iy)(x-iy)\\
        &=
        \left(\frac{a+ib}{1-c}\right)\left(\frac{a-ib}{1-c}\right)\\
        &=
        \frac{a^2+b^2}{(1-c)^2}\\
        &=
        \frac{1-c^2}{(1-c)^2}\\
        &=
        \frac{1+c}{1-c}\\
        &=
        -1+\frac{2}{1-c}
    \end{align*}
    which rearranges to
    \[
        c=
        \frac{x^2+y^2-1}{x^2+y^2+1}
    \]

    Then
    \[
        a+ib=
        \frac{2(x+iy)}{x^2+y^2+1}
    \]
    and we may compare real and imaginary parts for the required result.
\end{proof}

\begin{defn}[Extended complex plane]
    The \emph{extended complex plane} is the set $\comps\cup\{\infty\}$, which is denoted by $\tilde{\comps}$.
    Note that $\infty$ corresponds to $N\in S^2$.
\end{defn}

\begin{defn}[Riemann sphere]
    When we identify $S^2$ with $\tilde{\comps}$ using stereographic projection, $S^2$ is known as the \emph{Riemann sphere}.
\end{defn}

\begin{cor}[Antipodal points]
    If $M\in S^2$ corresponds to $z\in\tilde{\comps}$, then the antipodal point, $-M$, corresponds to $-1/\bar{z}$.
\end{cor}

\begin{proof}
    Say $M=(a,b,c)$, which corresponds to $z=(a+ib)/(1-c)$.
    Then $-M$ corresponds to
    \[
        w=
        \frac{-a-ib}{1+c}
    \]
    and
    \begin{align*}
        w\bar{z}
        &=
        \frac{(a-ib)(-a-ib)}{(1-c)(1+c)}\\
        &=
        \frac{-a^2-b^2}{1-c^2}\\
        &=
        \frac{c^2-1}{1-c^2}\\
        &=
        -1
    \end{align*}
\end{proof}

\begin{thm}
    Circles and lines in the extended complex plane correspond to circles on the Riemann sphere, and vice-versa.
    In particular, lines in the extended complex plane correspond to circles passing through $N$.
\end{thm}

\begin{proof}
    Consider the plane $\Pi$ with equation $Aa+Bb+Cc=D$.
    This plane will intersect with $S^2$ in a circle if $A^2+B^2+C^2>D^2$.
    The point corresponding to $z=x+iy$ lies on $\Pi$ iff
    \[
        2Ax+2By+C(x^2+y^2-1)=
        D(x^2+y^2+1)
    \]
    which can be rewritten as
    \[
        (C-D)(x^2+y^2)+2Ax+2By+(-C-D)=0
    \]
    This is the equation of a circle in $\tilde{\comps}$ if $C\neq D$.
    The centre is then
    \[
    \left(\frac{A}{D-C},\frac{B}{D-C}\right)
    \]
    and the radius is
    \[
    \frac{\sqrt{A^2+B^2+C^2-D^2}}{C-D}
    \]

    Further, all circles can be written like this: set $C-D=1$ and let $A,B,C+D$ vary arbitrarily.

    On the other hand, if $C=D$ then we have the equation
    \[
        Ax+By=C
    \]
    which is the equation of a line.
    Moreover, any line can be written in this form.
    Note that $C=D$ iff $N=(0,0,1)\in\Pi$.
\end{proof}

\begin{defn}[Circlines]
    A \emph{circline} is any subset of $\tilde{\comps}$ which is either a line or a circle.
\end{defn}

\begin{prop}
    Stereographic projection is conformal (that is, angle preserving).
\end{prop}

\begin{proof}
    Without loss of generality, we can consider the angle defined by the real axis and an arbitrary line meeting it at the point $p\in\reals$ and making an angle $\theta$.
    So points on the two lines can be parametrised as
    \[
        z=p+t,\qquad
        z=p+te^{i\theta}
    \]
    where $t$ is real.
    These points map onto the sphere as
    \[
        \vc{r}(t)=
        \frac{(2(p+t),0,(p+t)^2-1)}{1+(p+t)^2},\qquad
        \vc{s}(t)=
        \frac{(2(p+t\cos\theta),2t\sin\theta,(p+t\cos\theta)^2+t^2\sin^2\theta-1)}{1+(p+t)^2}
    \]

    We can calculate the tangent vectors at $0$ of each line by calculating $\vc{r}'(0)$ and $\vc{s}'(0)$.
    Then then angle $\phi$ between these two tangent vectors is given by
    \[
        \cos\phi=
        \frac{\vc{r}'(0)\cdot\vc{s}'(0)}{|\vc{r}'(0)||\vc{s}'(0)|}
    \]
    and, upon calculation, we see that
    \[
        \cos\phi=
        \cos\theta
    \]
    as required.
\end{proof}

\begin{defn}[M\"obius transformations]
    A \emph{M\"obius transformation} is a map $f:\tilde{\comps}\to\tilde{\comps}$ of the form
    \begin{equation}
        f(z)=
        \frac{az+b}{cz+d},\quad
        \text{where }ad\neq bc
    \end{equation}

    We define
    \[
        f(\infty)=
        \left\{
        \begin{array}{lr}
            \frac{a}{c} & \text{if }c\neq0\\
            \infty & \text{if }c=0
        \end{array}
        \right.
    \]
    and, if $c\neq0$,
    \[
        f(-\frac{d}{c})=
        \infty,
    \]
\end{defn}

\begin{prop}
    M\"obius transformations form a group of transformations $\tilde{\comps}\to\tilde{\comps}$ generated (under composition) by
    \begin{enumerate}[(i)]
        \item \emph{translations}:
        maps of the form $z\mapsto z+k$, where $k\in\comps$
        \item \emph{scalings} or \emph{dilations}:
        maps of the form $z\mapsto kz$, where $k\in\comps$
        \item \emph{inversions}:
        the map $z\mapsto 1/z$ (N.B. this may not always look like an `inversion'; the name can be misleading)
    \end{enumerate}
\end{prop}

\begin{proof}
    Note that, if $c\neq0$, then
    \[
        \frac{az+b}{cz+d}=
        \frac{a}{c}-\frac{bc-ad}{c^2z+cd}
    \]
    is a composition of various translations, scalings, and inversions.
    If $c=0$ then $d\neq0$, and clearly $z\mapsto(a/d)z+(b/d)$ is a composition of a scaling and a translation.

    This shows that M\"obius transformations are a subset of the group generated by translations, scalings, and inversions.
    It is also clear that these three types of transformations are all special types of M\"obius transformations.
    Finally, if $f(z)$ is a M\"obius transformation, then we can check that $f(z+k),f(kz)$, and $f(1/z)$ are all M\"obius transformations too.
\end{proof}

\begin{prop}
    M\"obius transformations are bijections from $\tilde{\comps}\to\tilde{\comps}$.
\end{prop}

\begin{proof}
    Only remains to prove that translations, scalings, and inversions are all bijections (when we include $\infty$).
\end{proof}

\begin{prop}
    Given two triples of distinct points, $z_1,z_2,z_3$ and $w_1,w_2,w_3$, both in $\tilde{\comps}$, there is a unique M\"obius transformation $f$ such that $f(z_i)=w_i$ for $i=1,2,3$.
\end{prop}

\begin{proof}
    Note that the map
    \[
        f(z)=
        \frac{(z_2-z_3)(z-z_1)}{(z_2-z_1)(z-z_3)}
    \]
    is a M\"obius transformation which maps $z_1,z_2,z_3$ to $0,1,\infty$, respectively.
    There is a similar transformation $g$ which maps $w_1,w_2,w_3$ to $0,1,\infty$, respectively.
    By the properties of M\"obius transformations, $g^{-1}f$ is a M\"obius transformation which maps each $z_i$ to $w_i$.

    To show uniqueness, suppose that $h$ is another such map.
    Then $ghf^{-1}$ is a M\"obius transformation which maps $0,1,\infty$ to $0,1,\infty$.
    If we write
    \[
        ghf^{-1}(z)=
        \frac{az+b}{cz+d}
    \]
    then $0\mapsto0$ means that $b=1$, $1\mapsto1$ means that $a+b=c+d$, and $\infty\mapsto\infty$ means that $c=0$.
    Hence $ghf^{-1}(z)=z$ for all $z$, and $h=g^{-1}f$, as required.
\end{proof}

\begin{prop}
    M\"obius transformations map circlines to circlines.
\end{prop}

\begin{proof}
    Let $A,C\in\reals,B\in\comps$.
    Then the circlines are the solutions sets of the equation
    \[
        Az\bar{z}+\bar{B}z+B\bar{z}+C=0
    \]
    as we have already shown.

    Inversion maps this to
    \[
        Cz\bar{z}+\bar{B}\bar{z}+Bz+a=0
    \]
    and it is clear that translations map circlines to circlines.
    Scaling in the form $z\mapsto kz$ maps the circline equation to
    \[
        Az\bar{z}+\bar{Bk}z+Bk\bar{z}+|k^2|C=0
    \]
    which is another circline.

    Hence, a M\"obius transformation, which is a composition of these maps, also maps circlines to circlines.
\end{proof}


\section{Conformal maps}

\begin{defn}[Conformal maps]
    A holomorphic map $f:U\to\comps$ is said to be \emph{conformal} if $f'(z)\neq0$ for all $z\in U$.
\end{defn}

\begin{prop}
    A conformal map is angle preserving and sense preserving.
\end{prop}

\begin{proof}
    Let $f:U\to\comps$ be holomorphic on a an open set $U$.
    Let $z_0\in U$, and $\gamma_1:[-1,1]\to U$ and $\gamma_2:[-1,1]\to U$ be two paths which meet at $z_0=\gamma_1(0)=\gamma_2(0)$.
    The original curves meet at $z_0$ with the (signed) angle
    \[
        \theta=
        \arg\gamma_2'(0)-\arg\gamma_1'(0)=
        \arg\frac{\gamma_2'(0)}{\gamma_1'(0)}
    \]

    The images of the curves, $f(\gamma_1)$ and $f(\gamma_2)$, meet at $f(z_0)$ at the angle
    \begin{align*}
        \phi
        &=
        \arg(f\circ\gamma_2)'(0)-\arg(f\circ\gamma_1)'(0)\\
        &=
        \arg\frac{(f\circ\gamma_2)'(0)}{(f\circ\gamma_1)'(0)}\\
        &=
        \arg\frac{f'(\gamma_2(0))\gamma_2'(0)}{f'(\gamma_1(0))\gamma_1'(0)}\\
        &=
        \arg\frac{f'(z_0)\gamma_2'(0)}{f'(z_0)\gamma_1'(0)}\\
        &=
        \arg\frac{\gamma_2'(0)}{\gamma_1'(0)}\\
        &=
        \theta
    \end{align*}
\end{proof}

\begin{defn}[Conformal equivalence]
    Two domains are \emph{conformally equivalent} if there exists a conformal bijection between them.
\end{defn}

\begin{rem}
    There are three main types of conformal which are used:
    \begin{enumerate}[(i)]
        \item \emph{M\"obius transformations}:
        We can check that M\"obius transformations are conformal maps.
        These maps are useful for conformally mapping regions bounded by circlines.
        \item \emph{Power maps}:
        Maps of the form $z\mapsto z^{\alpha}$ are conformal everywhere except at $0$.
        These are very useful for changing the angle at the origin (but obviously only when the origin is on the boundary of the region, and, crucially, not in it).
        \item \emph{Exponential}:
        The map $z\mapsto\exp z$ is particularly useful for mapping semi-infinite and infinite bars.
    \end{enumerate}
\end{rem}

\begin{rem}
    If a domain $U$ is bounded by two circline arcs or segments which meet at $\alpha$ and $\beta$, then the best thing to do first, almost without any further consideration, is to apply the map $(z-\alpha)/(z-\beta)$.
    This map sends $\alpha$ to $0$ and $\beta$ to $\infty$, and so takes the two circline arcs or segments to half lines meeting at the origin.
\end{rem}

\begin{thm}[Riemann Mapping Theorem]
    Let $U$ be a simply-connected domain with $U\neq\comps$.
    Then $U$ is conformally equivalent to $D(0,1)$.
    Further, if the boundary of $U$ is smooth, then the conformal equivalence can be extended between $U\cup\partial U$ and $D(0,1)$.
\end{thm}

\begin{proof}
    Once again, beyond the level of this course.
\end{proof}

\begin{prop}
    Let $U$ and $V$ be open subsets of $\comps$.
    Let $f:U\to V$ be holomorphic, and $\phi:V\to\reals$ be harmonic.
    Then $\phi\circ f$ is harmonic.
\end{prop}

\begin{proof}
    Let $f=u+iv$, and also define $\psi=\phi\circ f$.
    If $z=x+iy$ then we can write $\psi$ as
    \[
        \psi(x+iy)=
        \phi(u(x,y),v(x,y))
    \]

    As $\phi$ is harmonic we have that $\phi_{xx}+\phi_{yy}=0$, and as $f$ is holomorphic it satisfies the Cauchy-Riemann equations, and thus (as proven before) $u$ and $v$ are both harmonic too.
    Thus, and also by the chain rule,
    \[
        \psi_{xx}+\psi_{yy}=
        \phi_{uu}(u_x^2+u_y^2)+\phi_{vv}(v_x^2+v_y^2)
    \]

    Now, also by the Cauchy-Riemann equations,
    \[
        |f'(z)|^2=
        u_x^2+u_y^2=
        v_x^2+v_y^2
    \]
    and thus, as $\phi$ is harmonic,
    \[
        \psi_{xx}+\psi_{yy}=
        |f'(z)|^2(\phi_{uu}+\phi_{vv})=
        0
    \]
\end{proof}

\begin{defn}[Space of harmonic functions]
    Given an open subset $U\subseteq\comps$, we define $H(U)$ to be the vector space of harmonic functions on $U$.
\end{defn}

\begin{cor}
    Let $U$ and $V$ be conformally equivalent open subsets of $\comps$.
    Then $H(U)$ and $H(V)$ are isomorphic as vector spaces.
\end{cor}

\begin{proof}
    Let $f:U\to V$ be a conformal equivalence.
    This then induces the map
    \begin{align*}
        H(V)\to H(U)&\text{ given by }\phi\mapsto\phi\circ f\\
        H(U)\to H(V)&\text{ given by }\phi\mapsto\phi\circ f^{-1}
    \end{align*}
    which are clearly linear (in $\phi$) and inverses of one another.
\end{proof}

\end{document}
