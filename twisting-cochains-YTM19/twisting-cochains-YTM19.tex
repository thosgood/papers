\documentclass{beamer}

\usetheme[numbering=fraction,block=fill]{metropolis}

\usepackage{mathrsfs}

\title{Twisting cochains and twisted complexes}
\subtitle{Simplicial methods in complex-analytic algebraic geometry}
\author{Tim Hosgood}
\institute{Université d'Aix-Marseille}
\date{24/07/19}

\begin{document}
    \begin{frame}
        \titlepage
    \end{frame}

    \begin{frame}
        \frametitle{Plan}
        \tableofcontents
    \end{frame}


    \section{History}

        \begin{frame}\frametitle{First steps}
            \begin{itemize}
                \item Edgar H Brown. ``Twisted tensor products, I''. In: \emph{Annals of Mathematics} 69.1 (1959), pp.~223--246.
                \item John C Moore. ``Differential homological algebra''. In: \emph{Actes du Congres International des Mathématiciens} 1 (1970), pp.~335--339.
            \end{itemize}
        \end{frame}

        \begin{frame}\frametitle{Coherent sheaves}
            \begin{itemize}
                \item Domingo Toledo and Yue Lin L Tong. ``A parametrix for $\delta$ and Riemann-Roch in Čech theory''. In: \emph{Topology} 15.4 (1976), pp.~273--301.
                \item Domingo Toledo and Yue Lin L Tong. ``Duality and Intersection Theory in Complex Manifolds. I''. In: \emph{Mathematische Annalen} 237 (1978), pp.~41--77.
                \item Nigel R O'Brian, Domingo Toledo, and Yue Lin L Tong. ``The Trace Map and Characteristic Classes for Coherent Sheaves''. In: \emph{American Journal of Mathematics} 103.2 (1981), pp.~225--252.
            \end{itemize}
        \end{frame}

        \begin{frame}\frametitle{Triangulation and stability}
            \begin{itemize}
                \item A I Bondal and M M Kapranov. ``Enhanced Triangulated Categories''. In: \emph{Math. USSR Sbornik} 70.1 (1991), pp.~1--15.
                \item Giovanni Faonte. \emph{Simplicial nerve of an A-infinity category}. 2015. arXiv: 1312.2127 [math.AT].
            \end{itemize}
        \end{frame}


    \section{Twisting cochains (OTT)}

        \subsection{Bicomplexes}

            \begin{frame}\frametitle{Nice spaces}
                \begin{definition}[Stein spaces]
                    A complex-analytic\footnote{analytic = $\mathcal{O}_Y$ is holomorphic functions, $Y$ has the $\mathbb{C}^n$-induced topology; algebraic = $\mathcal{O}_Y$ is algebraic functions, $Y$ has the Zariski topology.} manifold $Y$ is said to be \emph{Stein} if it is
                    \begin{enumerate}
                        \item \emph{holomorphically convex}; and
                        \item \emph{holomorphically separable}.
                    \end{enumerate}
                \end{definition}

                \pause

                \begin{block}{Motto}
                    Stein things are nice.
                \end{block}

                \pause

                Throughout, $X$ is a complex-analytic manifold with a nice\footnote{\emph{Locally finite}, \emph{Stein}, and \emph{trivialising} (for the bundles in question).} cover $\mathcal{U}=\{U_\alpha\}_{\alpha\in I}$.
            \end{frame}

            \begin{frame}\frametitle{Endomorphisms of bounded-graded modules}
                Let $V=\{V_\alpha^\bullet\}$ be a collection of \emph{bounded-graded $\mathcal{O}_{U_\alpha}$-modules}:
                \begin{equation*}
                    V_\alpha^\bullet = \bigoplus_{q\in\mathbb{N}}V_\alpha^q\quad\text{such that }V_\alpha^q\text{ is zero for all but finitely many }q.
                \end{equation*}

                \pause

                Think of a bounded chain complex of vector bundles, but without the information of a differential.

                \pause

                \begin{definition}[Endomorphisms]
                    The collection of \emph{degree-$q$ endomorphisms $\mathrm{End}^q(V)$ of $V$} is, over each $U_{\alpha_0\ldots\alpha_p}$, given by
                    \begin{equation*}
                        \mathrm{End}^q(V)|U_{\alpha_0\ldots\alpha_p} = \bigoplus_{i\in\mathbb{Z}}\mathrm{Hom}\big( V_{\alpha_p}^i|U_{\alpha_0\ldots\alpha_p}, V_{\alpha_0}^{i+q}|U_{\alpha_0\ldots\alpha_p} \big).
                    \end{equation*}
                \end{definition}

                \pause

                \begin{alertblock}{Warning}
                    The maps are from the $\alpha_p$ part to the $\alpha_0$ part.
                \end{alertblock}
            \end{frame}

            \begin{frame}\frametitle{The bicomplex}
                % Given a (chain) complex $(K^\circ,d)$ of vector bundles on $X$, define the bicomplex $(\hat{\mathscr{C}}^\bullet(\mathcal{U},K^\circ),\hat{\delta},d)$ by
                % \begin{equation*}
                %     \hat{\mathscr{C}}^p(\mathcal{U},K^q) = 
                % \end{equation*}
            \end{frame}

        \subsection{Maurer-Cartan}


    \section{Twisted complexes (BK)}
        
        \subsection{Pretriangulated vs. triangulated}
        
        \subsection{Generalisation of twisting cochains}
    

    \section{Other fun things}

        \subsection{The A-infinity Yoneda embedding}
        
        \subsection{The bar construction}
        
        \subsection{Simplicial nerve}

\end{document}
