\documentclass{beamer}

\usetheme[numbering=fraction,block=fill]{metropolis}

\usepackage{mathrsfs}
\usepackage{booktabs}

\title{Twisting cochains and twisted complexes}
\subtitle{Simplicial methods in complex-analytic geometry}
\author{Tim Hosgood}
\institute{Université d'Aix-Marseille\\\url{https://thosgood.github.io}}
\date{24/07/19}

\begin{document}
    \begin{frame}
        \titlepage
    \end{frame}

    \begin{frame}
        \frametitle{Plan}
        \tableofcontents
    \end{frame}


    \section{History}

        \begin{frame}\frametitle{First steps}
            \begin{itemize}
                \item Edgar H Brown. ``Twisted tensor products, I''. In: \emph{Annals of Mathematics} 69.1 (1959), pp.~223--246.
                \item John C Moore. ``Differential homological algebra''. In: \emph{Actes du Congres International des Mathématiciens} 1 (1970), pp.~335--339.
            \end{itemize}
        \end{frame}

        \begin{frame}\frametitle{Coherent sheaves}
            \begin{itemize}
                \item Domingo Toledo and Yue Lin L Tong. ``A parametrix for $\delta$ and Riemann-Roch in Čech theory''. In: \emph{Topology} 15.4 (1976), pp.~273--301.
                \item Domingo Toledo and Yue Lin L Tong. ``Duality and Intersection Theory in Complex Manifolds. I''. In: \emph{Mathematische Annalen} 237 (1978), pp.~41--77.
                \item Nigel R O'Brian, Domingo Toledo, and Yue Lin L Tong. ``The Trace Map and Characteristic Classes for Coherent Sheaves''. In: \emph{American Journal of Mathematics} 103.2 (1981), pp.~225--252.
            \end{itemize}
        \end{frame}

        \begin{frame}\frametitle{Triangulation and stability}
            \begin{itemize}
                \item A I Bondal and M M Kapranov. ``Enhanced Triangulated Categories''. In: \emph{Math. USSR Sbornik} 70.1 (1991), pp.~1--15.
                \item Giovanni Faonte. \emph{Simplicial nerve of an A-infinity category}. 2015. arXiv: 1312.2127 [math.AT].
            \end{itemize}
        \end{frame}


    \section{Twisting cochains (OTT)}

        \subsection{The bicomplex}

            \begin{frame}\frametitle{Nice spaces}
                \begin{definition}[Stein spaces]
                    A complex-analytic\footnote{analytic = $\mathcal{O}_Y$ is holomorphic functions, $Y$ has the $\mathbb{C}^n$-induced topology; algebraic = $\mathcal{O}_Y$ is algebraic functions, $Y$ has the Zariski topology.} manifold $Y$ is said to be \emph{Stein} if it is
                    \begin{enumerate}
                        \item \emph{holomorphically convex}; and
                        \item \emph{holomorphically separable}.
                    \end{enumerate}
                \end{definition}

                \pause

                \begin{block}{Motto}
                    Stein things are nice.
                \end{block}

                \pause

                Throughout, $X$ is a complex-analytic manifold with a nice\footnote{\emph{Locally finite}, \emph{Stein}, and \emph{trivialising} (for the bundles in question).} cover $\mathcal{U}=\{U_\alpha\}_{\alpha\in I}$.
            \end{frame}

            \begin{frame}\frametitle{Endomorphisms of bounded-graded modules}
                Let $V=\{V_\alpha^\bullet\}$ be a collection of \emph{bounded-graded $\mathcal{O}_{U_\alpha}$-modules}:
                \begin{equation*}
                    V_\alpha^\bullet = \bigoplus_{q\in\mathbb{N}}V_\alpha^q\quad\text{such that }V_\alpha^q\text{ is zero for all but finitely many }q.
                \end{equation*}

                \pause

                Think of a bounded chain complex of vector bundles, but without the information of a differential.

                \pause

                \begin{definition}[Endomorphisms]
                    The collection of \emph{degree-$q$ endomorphisms $\mathrm{End}^q(V)$ of $V$} is, over each $U_{\alpha_0\ldots\alpha_p}$, given by
                    \begin{equation*}
                        \mathrm{End}^q(V)|U_{\alpha_0\ldots\alpha_p} = \bigoplus_{i\in\mathbb{Z}}\mathrm{Hom}\big( V_{\alpha_p}^i|U_{\alpha_0\ldots\alpha_p}, V_{\alpha_0}^{i+q}|U_{\alpha_0\ldots\alpha_p} \big).
                    \end{equation*}
                \end{definition}
            \end{frame}

            \begin{frame}\frametitle{Source and target}
                \begin{alertblock}{Warning}
                    The maps are from the $\alpha_p$ part to the $\alpha_0$ part.
                \end{alertblock}

                \pause

                We discuss this later.
            \end{frame}

            \begin{frame}\frametitle{The deleted Čech complex}
                \begin{definition}[Deleted Čech complex]
                    Define the chain complex $(\hat{\mathscr{C}}^\bullet(\mathcal{U},\mathrm{End}^\circ(V)),\hat{\delta})$ by
                    \begin{equation*}
                        \hat{\mathscr{C}}^p\big(\mathcal{U},\mathrm{End}^q(V)\big) = \bigoplus_{(\alpha_0,\ldots,\alpha_p)} \mathrm{End}^q(V)|U_{\alpha_0\ldots\alpha_p}
                    \end{equation*}
                    (where $\mathrm{End}^q(V)|U_{\alpha_0\ldots\alpha_p}=0$ if $U_{\alpha_0\ldots\alpha_p}=\varnothing$) with the \emph{\textbf{deleted} Čech differential}
                    \begin{gather*}
                        \hat{\delta} \colon \hat{\mathscr{C}}^p\big(\mathcal{U},\mathrm{End}^q(V)\big) \to \hat{\mathscr{C}}^{p+1}\big(\mathcal{U},\mathrm{End}^q(V)\big)\\
                        (\hat{\delta}c)_{\alpha_0\ldots\alpha_{p+1}} = \sum_{i=1}^p (-1)^i c_{\alpha_0\ldots\widehat{\alpha_i}\ldots\alpha_{p+1}}.
                    \end{gather*}
                \end{definition}
            \end{frame}

            \begin{frame}\frametitle{A notational note}
                We use $\hat{\mathscr{C}}$ and $\hat{\delta}$ for the \emph{deleted} Čech objects and $\check{\mathscr{C}}$ and $\check{\delta}$ for the `full' Čech objects.
            \end{frame}

            \begin{frame}\frametitle{Further structure}
                \begin{itemize}
                    \item If $V$ has a differential then this gives us a \emph{bicomplex}.
                    \pause
                    \item There is a natural multiplication structure given by composition:
                        \begin{equation*}
                            (c^{p,q}\cdot \tilde{c}^{\tilde{p},\tilde{q}})_{\alpha_0\ldots\alpha_{p+\tilde{p}}} = (-1)^{q\tilde{p}} c_{\alpha_0\ldots\alpha_p}^{p,q} \tilde{c}_{\alpha_p\ldots\alpha_{p+\tilde{p}}}^{\tilde{p},\tilde{q}}.
                        \end{equation*}
                    \pause
                    \item We could define the same complex for an arbitrary bounded graded vector bundle, i.e.
                    \begin{equation*}
                        \hat{\mathscr{C}}^p(\mathcal{U},V^q) = \bigoplus_{(\alpha_0,\ldots,\alpha_p)} V_{\alpha_0}^q
                    \end{equation*}
                    but where the deleted Čech differential only omits the \emph{first} index (but includes the $(p+1)$th).
                \end{itemize}
            \end{frame}

            \begin{frame}\frametitle{A brief interlude on bundles}
                A holomorphic vector bundle $E$ on $X$ is described exactly by its \emph{transition maps $g_{\alpha\beta}\in\mathrm{GL}(n,\mathbb{C})$}, which describe the change in trivialisation from over $U_\beta$ to over $U_\alpha$.

                \pause

                These transition maps satisfy two conditions:

                \pause
                
                \begin{enumerate}
                    \item $g_{\alpha\beta}g_{\beta\gamma}=g_{\alpha\beta}$ (the \emph{cocycle} condition); and
                    \pause
                    \item $g_{\alpha\alpha}=\mathrm{id}$ (the \emph{invertibility} condition).
                \end{enumerate}

                \pause

                Note that these are maps from $E|U_{\alpha_p}$ to $E|U_{\alpha_0}$ in the specific case where $p=1$.
            \end{frame}

            \begin{frame}\frametitle{Rewriting the cocycle condition}
                Thinking of $g_{\alpha\beta}$ as an element of $\hat{\mathscr{C}}^1(\mathcal{U},E)$, we see that
                \begin{gather*}
                    (\hat{\delta}g)_{\alpha\beta\gamma} = -g_{\alpha\gamma}\\
                    (g\cdot g)_{\alpha\beta\gamma} = g_{\alpha\beta}g_{\beta\gamma}.
                \end{gather*}

                \pause

                This means that we can rewrite the cocycle condition as
                \begin{equation*}
                    \hat{\delta}g + g\cdot g = 0,
                \end{equation*}
                which looks like the \emph{Maurer-Cartan equation} (an observation to which we will later return).
            \end{frame}

            \begin{frame}\frametitle{Twisting cochains}
                \begin{definition}[Twisting cochains]
                    A \emph{(holomorphic) twisting cochain over $V$} is a formal sum
                    \begin{equation*}
                        \mathrm{a} = \bigoplus_{k\in\mathbb{N}} \mathrm{a}^{k,1-k}
                    \end{equation*}
                    where $\mathrm{a}^{k,1-k}\in\hat{\mathscr{C}}^k(\mathcal{U},\mathrm{End}^{1-k}(V))$ such that
                    \begin{enumerate}
                        \item $\hat{\delta}\mathrm{a} + \mathrm{a}\cdot\mathrm{a} = 0$; and
                        \item $\mathrm{a}_{\alpha\alpha}^{1,0}=\mathrm{id}$.
                    \end{enumerate}
                \end{definition}

                \pause

                The invertibility condition ``should'' really be weakened by asking only that $\mathrm{a}_{\alpha\alpha}^{1,0}$ be \emph{homotopic} to the identity.
            \end{frame}

            \begin{frame}\frametitle{Twisting cochains (cont.)}
                \begin{alertblock}{Warning}
                    The multiplication is \textbf{not} simply component-wise: it is given by taking all possible combinations, i.e.
                    \begin{equation*}
                        (\mathrm{a}\cdot\mathrm{b})^{p,s} = \bigoplus_{\substack{q+q'=p\\t+t'=s}} \mathrm{a}^{q,t}\cdot\mathrm{b}^{q',t'}.
                    \end{equation*}
                \end{alertblock}

                \pause
                
                \begin{itemize}
                    \item It might be the case that all but finitely many of the $\mathrm{a}^{k,1-k}$ are zero, but \textbf{never} $\mathrm{a}^{1,0}$, since it has to be the identity on $\alpha\alpha$.
                    \pause
                    \item If $V$ has a differential then $\mathrm{a}$ is an element of total degree $1$.
                    \pause
                    \item We haven't said \emph{when} twisting cochains exist, but under pretty mild assumptions they always do (by an inductive construction).
                \end{itemize}
            \end{frame}

            \begin{frame}\frametitle{Unpacking the definition}
                \begin{description}
                    \item[($k=0$) $\rightsquigarrow$] $\mathrm{a}_{\alpha}^{0,1}\cdot\mathrm{a}_{\alpha}^{0,1}=0$, which tells us that $\mathrm{a}_{\alpha}^{0,1}$ is a \emph{differential on $V_\alpha^\bullet$}.
                    \pause
                    \item[($k=1$) $\rightsquigarrow$] $\mathrm{a}_{\alpha}^{0,1}\cdot\mathrm{a}_{\alpha\beta}^{1,0} = \mathrm{a}_{\alpha\beta}^{1,0}\cdot\mathrm{a}_{\beta}^{0,1}$, which tells us that we have a \emph{chain map of chain complexes}
                        \begin{equation*}
                            \mathrm{a}_{\alpha\beta}^{1,0} \colon \big(V_\beta^\bullet|U_{\alpha\beta},\mathrm{a}_\beta^{0,1}\big) \to \big(V_\alpha^\bullet|U_{\alpha\beta},\mathrm{a}_\alpha^{0,1}\big)
                        \end{equation*}
                    \vspace{-1.5em}
                    \pause
                    \item[($k=2$) $\rightsquigarrow$] $-\mathrm{a}_{\alpha\gamma}^{1,0} + \mathrm{a}_{\alpha\beta}^{1,0}\cdot\mathrm{a}_{\beta\gamma}^{1,0} = \mathrm{a}_{\alpha}^{0,1}\cdot\mathrm{a}_{\alpha\beta\gamma}^{2,-1} + \mathrm{a}_{\alpha\beta\gamma}^{2,-1}\cdot\mathrm{a}_{\gamma}^{0,1}$, which says that $\mathrm{a}_{\alpha\beta\gamma}^{2,-1}$ witnesses a \emph{chain homotopy} between $\mathrm{a}_{\alpha\gamma}^{1,0}$ and $\mathrm{a}_{\alpha\beta}^{1,0}\cdot\mathrm{a}_{\beta\gamma}^{1,0}$.
                        On $\alpha\beta\alpha$ and $\beta\alpha\beta$ this tells us that $\mathrm{a}_{\alpha\beta}^{1,0}$ and $\mathrm{a}_{\beta\alpha}^{1,0}$ are \emph{chain homotopic inverses}, i.e. \emph{quasi-isomorphism}.
                \end{description}
            \end{frame}

            \begin{frame}\frametitle{Unpacking the definition (cont.)}
                \begin{description}
                    \item[($k\geqslant3$) $\rightsquigarrow$] some sort of `higher homotopic gluings', whatever this might mean.
                \end{description}

                \pause

                This is one of the things that we want to formalise!

                \pause

                \begin{block}{Extra-curricular}
                    By taking (internal) homology we obtain something strict: a complex of \emph{coherent sheaves} $\mathrm{H}^\bullet(\mathrm{a})$.
                    This is because quasi-isomorphisms become strict isomorphisms in homology.

                    We can use this fact to construct twisting cochains that resolve coherent sheaves by taking \emph{local} resolutions by vector bundles.
                \end{block}
            \end{frame}

        \subsection{The total complex}

            \begin{frame}\frametitle{The total differential}
                \begin{lemma}
                    For any $\mathrm{a}\in\mathrm{Tot}^1\hat{\mathscr{C}}^\bullet(\mathcal{U},\mathrm{End}^\circ(V))$, the map
                    \begin{align*}
                        \mathrm{D}_\mathrm{a} \colon \mathrm{Tot}^r\hat{\mathscr{C}}^\bullet(\mathcal{U},V^\circ) &\to \mathrm{Tot}^{r+1}\hat{\mathscr{C}}^\bullet(\mathcal{U},V^\circ)\\
                        c &\mapsto \hat{\delta}c+c\cdot\mathrm{a}
                    \end{align*}
                    defines a differential (i.e. squares to zero) if and only if $\mathrm{a}$ is a twisting cochain.
                \end{lemma}
                \begin{proof}
                    (Tedious) definition chasing.
                \end{proof}
            \end{frame}

            \begin{frame}\frametitle{The total differential (cont.)}
                We can actually define twisting cochains in a different way using this lemma (but we won't do so today).

                \pause

                But this approach lets us think of a twisting cochain as a \emph{first-order perturbation of the deleted Čech differential}.
            \end{frame}

            \begin{frame}\frametitle{Examples}
                \begin{example}
                    Look at the most trivial example: let $V$ be an \emph{ungraded} vector bundle, and $\mathrm{a} = \mathrm{a}^{0,1}+\mathrm{a}^{1,0}$, where $\mathrm{a}_{\alpha}^{0,1} = \mathrm{id}_{V_\alpha}$, and the $\mathrm{a}^{1,0}$ are the transition maps.
                    Then
                    \begin{equation*}
                        (\mathrm{D}_\mathrm{a}c)_{\alpha_0\ldots\alpha_{p+1}} = \mathrm{a}_{\alpha_0\alpha_1}^{1,0}c_{\alpha_1\ldots\alpha_{p+1}} + \sum_{i=1}^{p+1}(-1)^i c_{\alpha_0\ldots\widehat{\alpha_i}\ldots\alpha_{p+1}}.
                    \end{equation*}

                    \pause

                    Note that we couldn't use the full Čech differential on $\hat{\mathscr{C}}^\bullet(\mathcal{U},V^\circ)$ because everything has to lie over $U_{\alpha_0}$, but this total differential solves that problem --- recall that $\mathrm{a}_{\alpha_0\alpha_1}^{1,0}$ is a (quasi-)isomorphism.

                    \pause

                    A spectral-sequence argument shows that, in fact, $\mathrm{D}_\mathrm{a}$ here really is `the same as' the full Čech differential.
                \end{example}
            \end{frame}

            \begin{frame}\frametitle{Examples (cont.)}
                \begin{example}
                    Now look at a slightly-less trivial example: let $V^\bullet$ consist of \emph{complexes} $(V_\alpha^\bullet,\mathrm{d}_\alpha)$ of vector bundles, and $\mathrm{a} = \mathrm{a}^{0,1}+\mathrm{a}^{1,0}$, where $\mathrm{a}_{\alpha}^{0,1} = \mathrm{d}_\alpha$, and the $\mathrm{a}^{1,0}$ are the transition maps.
                    Then
                    \begin{align*}
                        (\mathrm{D}_\mathrm{a}c)_{\alpha_0\ldots\alpha_{p+1}} &= (-1)^p\mathrm{a}_{\alpha_0}^{0,1}c_{\alpha_0\ldots\alpha_p} + \mathrm{a}_{\alpha_0\alpha_1}^{1,0}c_{\alpha_1\ldots\alpha_{p+1}}\\
                        &\qquad+\sum_{i=1}^{p+1}(-1)^i c_{\alpha_0\ldots\widehat{\alpha_i}\ldots\alpha_{p+1}}.
                    \end{align*}

                    \pause
                    
                    Identifying the second and third terms with the full Čech differential, as above, gives the usual total differential of the Čech bicomplex:
                    \begin{equation*}
                        \mathrm{D}_\mathrm{a} = \check{\delta} \pm \mathrm{d}_V.
                    \end{equation*}
                \end{example}
            \end{frame}

            \begin{frame}\frametitle{Why this emphasis on the first index?}
                \begin{itemize}
                    \item Transition maps naturally go from $\alpha_p$ to $\alpha_0$.
                    \pause
                    \item We want to be able to compare local things, and we need to pull everything back to lie over the same open set in order to do so.
                \end{itemize}
            \end{frame}

        \subsection{Maurer-Cartan}

            \begin{frame}\frametitle{The Maurer-Cartan equation}
                \begin{table}
                    \begin{tabular}{p{1.8cm}cp{3.9cm}}\toprule
                        Subject & Equation & Interpretation\\\midrule
                        Differential geometry & $F_\nabla = \mathrm{d}A + A\cdot A$ & curvature of a Koszul connection\footnote{Here be Christoffel symbols.}\\[.5em]
                        Gauge theory & $\Omega = \mathrm{d}A + \frac12[A\wedge A]$ & curvature of a principal connection\\[.5em]
                        Deformation theory & $\partial a+\frac12[a,a]$ & deformations of f.d. associative $k$-algebras with unit\footnote{There is the beautiful fact (that we won't explain at all) that $\mathrm{MC}(A\otimes\mathfrak{g}) \simeq \mathrm{Hom}_\mathsf{dg-alg}(\mathrm{CE}(\mathfrak{g}),A)$.}
                        \\\bottomrule
                    \end{tabular}
                \end{table}
            \end{frame}

            \begin{frame}\frametitle{Flatness}
                \begin{block}{Motto}
                    Solutions to (i.e. zeros of) the Maurer-Cartan equation are always (in some sense) \emph{flat objects}.
                \end{block}
            \end{frame}


    \section{Twisted complexes (BK)}
        
        \subsection{Pretriangulated vs. triangulated}
        
        \subsection{Generalisation of twisting cochains}
    

    \section{Other fun things}

        \subsection{The A-infinity Yoneda embedding}
        
        \subsection{The bar construction}
        
        \subsection{Simplicial nerve}

\end{document}
