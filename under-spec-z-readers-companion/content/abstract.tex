% !TEX root = ../under-spec-z.tex

\begin{abstract}
    Relative algebraic geometry is an approach to algebraic geometry using category theory.
    This allows us to generalise algebraic geometry to many different settings.
    This project will cover basic notions from category theory, symmetric monoidal categories, Grothendieck topologies, algebraic geometry relative to a symmetric monoidal category and the example of classical algebraic geometry and monoid algebraic geometry which is a version of the field with one element.

    One very important paper in this area is \cite{Toen:2005wxa}, which is written in French, and translating the first few sections of it into English would open this paper up to a whole new audience.
    Although mathematical French is in general not entirely impenetrable when one is armed with a good dictionary or glossary, a lot of the language found in this paper is hard to find in other sources.
    Further, when we are dealing with such abstract mathematics, grammar and semantics are of the utmost importance, and small variations can change the meaning wildly, making `on-the-fly' translation tricky.

    The aims of this project are: to translate the first few sections (those dealing with establishing the formalities of the subject) of \cite{Toen:2005wxa} into English; to provide ample editorial commentary concerning the translation and historical context; and to comment on the mathematics in the paper, providing enough background information for the new reader to be able to follow the main ideas -- the main emphasis is placed on this last point.
\end{abstract}
