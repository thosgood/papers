% !TEX root = ../under-spec-z.tex

\numberwithin{equation}{subsubsection}

\declaretheoremstyle[
    spacebelow=2\topsep,
    spaceabove=2\topsep,
    headfont=\normalfont\bfseries,
    bodyfont=\itshape,
    postheadspace=\newline,
    qed=${\lrcorner}$,
    headpunct={},
    notebraces={[}{]}
]{breakit}

\declaretheoremstyle[
    spacebelow=2\topsep,
    spaceabove=2\topsep,
    headfont=\normalfont\bfseries,
    bodyfont=\normalfont,
    postheadspace=\newline,
    qed=${\lrcorner}$,
    headpunct={},
    notebraces={[}{]}
]{breakup}

\declaretheorem[numberlike=equation,style=breakit]{theorem}
\declaretheorem[numberlike=theorem,style=breakit]{lemma}
\declaretheorem[numberlike=theorem,style=breakit]{corollary}

\declaretheorem[numberlike=theorem,style=breakup]{definition}
\declaretheorem[numberlike=theorem,style=breakup]{example}
\declaretheorem[numberlike=theorem,style=breakup]{note}

\numberwithin{theorem}{subsubsection}
\numberwithin{lemma}{subsubsection}
\numberwithin{corollary}{subsubsection}

\numberwithin{definition}{subsubsection}
\numberwithin{example}{subsubsection}
\numberwithin{note}{subsubsection}
